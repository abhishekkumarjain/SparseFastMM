%% This is file `sample-manuscript.tex', generated with the docstrip
%% utility.  The original source files were: samples.dtx (with
%% options: `manuscript') IMPORTANT NOTICE: For the copyright see the
%% source file.  Any modified versions of this file must be renamed
%% with new filenames distinct from sample-manuscript.tex.  For
%% distribution of the original source see the terms for copying and
%% modification in the file samples.dtx.  This generated file may be
%% distributed as long as the original source files, as listed above,
%% are part of the same distribution. (The sources need not
%% necessarily be in the same archive or directory.)  The first
%% command in your LaTeX source must be the \documentclass command.
\documentclass[manuscript,screen]{acmart}
\usepackage{bm}
\usepackage{lineno}
\linenumbers
% Paolo's definition. Yes, I have my way to define stuff. Please take
% a look



% Always include hyperref last
%\usepackage[bookmarks=true,breaklinks=true,letterpaper=true,colorlinks,linkcolor=black,citecolor=blue,urlcolor=black]{hyperref}
%\graphicspath{{../python/PNG/}}

\def\firmopsa{d}
\def\firmopsb{d}
\newcommand\AssignTags{TagIt}


% Definitions
% -----------
\def\x{{\mathbf x}}
\def\L{{\cal L}}
\def\spiral{SPIRAL\xspace}

\newcommand{\pred}{{}}

\newcommand{\tensor}[0]{\otimes}
\newcommand{\inv}[1]{{1{/}{#1}}}
%\newcommand{\inv}[1]{$\textstyle\frac{1}{#1}$}
\newcommand{\DFT}{\operatorname{\bf DFT}}
\newcommand{\WHT}{\operatorname{\bf WHT}}
\newcommand{\FIR}{\operatorname{\bf FIR}}
\newcommand{\one}[0]{{I}}
\newcommand{\eq}[0]{{=}}
\newcommand{\Tensor}[2]{#1{\tensor}#2}
\newcommand{\SPw}[2]{{\prec}#1,#2{\succ}}
\newcommand{\SPr}[2]{{<}#1,#2{>}}

\newcommand{\WTensorI}[2]{\Tensor{\WHT_{#1}}{\one_{#2}}}
\newcommand{\ITensorW}[2]{\Tensor{\one_{#1}}{\WHT_{#2}}}
\newcommand{\Wht}[2]{(\WTensorI{\WHT_{#1}}{#2})(\ITensorW{#1}{#2})}

\newcommand{\Times}[2]{{#1}{\times}{#2}}
\newcommand{\EQ}{{=}}
\newcommand{\Space}{{ S}}
\newcommand{\PEQ}{{{+}{=}}}
\newcommand{\SEQ}{{{-}{=}}}
\newcommand{\h}{\psi}
\newcommand{\balpha}{\bm{\alpha}}
\newcommand{\bgamma}{\bm{\gamma}}
\newcommand{\bbeta}{\bm{\beta}}
\newcommand{\bmu}{\bm{\mu}}
\newcommand{\bsigma}{\bm{\sigma}}


\newcommand{\f}{\varphi}
\newcommand{\Z}{\mathbb{Z}}
\newcommand{\A}{\mathbb{A}}
\newcommand{\B}{{\bf B}}
\newcommand{\X}{\mathbb{X}}
\newcommand{\T}{\mathbb{T}}
%\newcommand{\Q}{\mathbb{Q}}
\newcommand{\N}{\mathbb{N}}
\newcommand{\R}{\mathbb{R}}
\newcommand{\CC}{{\Gamma}}
\newcommand{\D}{\mathbb{D}}
\newcommand{\M}{\mathbb{M}}
\newcommand{\F}{\mathbb{F}}
\newcommand{\ST}{\mathbb{S}}
\newcommand{\Vc}[1]{{\boldsymbol #1}}
\newcommand{\fl}[1]{{\lfloor {#1} \rfloor}}
\newcommand{\cl}[1]{{\lceil {#1} \rceil}}
\newcommand{\half}[1]{{\frac{#1}{2}}}
\newcommand{\q}[1]{\Times{\cl{#1}}{\cl{#1}}}
\newcommand{\qq}[1]{\Times{\cl{#1}}{\fl{#1}}}
\newcommand{\qqq}[1]{\Times{\fl{#1}}{\cl{#1}}}
\newcommand{\qqqq}[1]{\Times{\fl{#1}}{\fl{#1}}}
\newcommand{\size}[1]{\sigma(#1)}
\newcommand{\Size}[3]{\sigma(\Vc{#1})\EQ\Times{#2}{#3}}
\newcommand{\n}{(\frac{n}{2})}
\newcommand{\m}{(\frac{n-1}{2})}
\newcommand{\Ceil}[1]{{\cl{\half{#1}}}}
\newcommand{\Floor}[1]{{\fl{\half{#1}}}}

\newcommand{\Endash}{{--}}
\newcommand{\Emdashbegin}{{---}}
\newcommand{\Emdashend}{{---}~}

\newcommand{\Transducer}[1] {{\mathbf #1 }}


\newcommand{\AO}  {\Times{\Ceil{m}}{\Ceil{n}}}
\newcommand{\AI} {\Times{\Ceil{m}}{\Floor{n}}}
\newcommand{\AII}{\Times{\Floor{m}}{\Ceil{n}}}
\newcommand{\AIII} {\Times{\Floor{m}}{\Floor{n}}}
\newcommand{\BO}  {\Times{\Ceil{n}}{\Ceil{p}}}
\newcommand{\BI} {\Times{\Ceil{n}}{\Floor{p}}}
\newcommand{\BII}{\Times{\Floor{n}}{\Ceil{p}}}
\newcommand{\BIII} {\Times{\Floor{n}}{\Floor{p}}}
\newcommand{\CO}  {\Times{\Ceil{m}}{\Ceil{p}}}
\newcommand{\CI} {\Times{\Ceil{m}}{\Floor{p}}}
\newcommand{\CII}{\Times{\Floor{m}}{\Ceil{p}}}
\newcommand{\CIII} {\Times{\Floor{m}}{\Floor{p}}}

\newcommand{\I}[2]{\Times{\Ceil{#1}}{\Ceil{#2}}}
\newcommand{\II}[2]{\Times{\Ceil{#1}}{\Floor{#2}}}
\newcommand{\III}[2]{\Times{\Floor{#1}}{\Ceil{#2}}}
\newcommand{\IV}[2]{\Times{\Floor{#1}}{\Floor{#2}}}


\newcommand{\Q}[2]{\Vc{#1}_{#2}}

\newcommand{\QAO}{{\Vc{A}_{0}}}
\newcommand{\QAI}{\Vc{A}_{1}}
\newcommand{\QAII}{\Vc{A}_{2}}
\newcommand{\QAIII} {\Vc{A}_{3}}

\newcommand{\QBO}  {\Vc{B}_{0}}
\newcommand{\QBI} {\Vc{B}_{1}}
\newcommand{\QBII}{\Vc{B}_{2}}
\newcommand{\QBIII} {\Vc{B}_{3}}

\newcommand{\QCO}  {\Vc{C}_{0}}
\newcommand{\QCI} {\Vc{C}_{1}}
\newcommand{\QCII}{\Vc{C}_{2}}
\newcommand{\QCIII} {\Vc{C}_{3}}
\newcommand{\FIGsize}[2]{{width={#1}, height={#2}}}

\newcommand{\Mod}[2]{{#1}{\text{ \bf mod }}{#2}}

\newcommand{\TableRef}[1]{Table \ref{#1}}
\newcommand{\FigureRef}[1]{Figure \ref{#1}}
\newcommand{\SectionRef}[1]{Section \ref{#1}}
\newcommand{\EquationRef}[1]{Equation \ref{#1}}
\newcommand{\OnPageRef}[1]{on page \pageref{#1}}

% Formatting
% ----------
%\setlength{\evensidemargin}{0mm}
%\setlength{\oddsidemargin}{0mm}
%\setlength{\textwidth}{6.5in}
%\setlength{\textheight}{9.5in}
%\setlength{\topmargin}{0in}
%\setlength{\headheight}{0in}
%\setlength{\headsep}{0mm}

\newenvironment{noinds_itemize}{\begin{list}{$\bullet$}
{\setlength{\rightmargin}{0em}
\setlength{\leftmargin}{1.2em}
\setlength{\itemsep}{0em}
\setlength{\topsep}{0em}
\setlength{\parsep}{0em}}}{\end{list}}



\newcommand{\mypar}[1]{{\bf #1.}}
\newcommand{\mytag}[2]{[#1]_{#2}}
\sloppy

%% jan 01, 2009 - dasdan - removed 'final' pdflatex cannot work with it.
%\usepackage[final=true,bookmarks=true,bookmarkstype=toc,colorlinks=true, linkcolor=blue,citecolor=blue]{hyperref}
%\usepackage[bookmarks=true,bookmarkstype=toc,colorlinks=true, linkcolor=blue,citecolor=blue]{hyperref}

% width fraction, mflops, relative, caption, label
\newcommand{\doublefigure}[5]{{\begin{figure}[htb]%
\centering %
\includegraphics[width=#1\linewidth]{#2}
\includegraphics[width=#1\linewidth]{#3}
\caption{#4}%bio                 
\label{#5}%
\end{figure}}}

\newcommand{\Doublefigure}[5]{{\begin{figure*} %
\centering %
\includegraphics[width=#1\linewidth]{#2}%
\includegraphics[width=#1\linewidth]{#3}%
\caption{#4}%
\label{#5}%
\end{figure*}}}

\newcommand{\Singlefigure}[4]{{\begin{figure*} %
\centering %
\includegraphics[width=#1\linewidth]{#2}%\figurebox{#1\linewidth}{}{}[#2]
\caption{#3}%
\label{#4}%
\end{figure*}}}

\newcommand{\singlefigure}[4]{{\begin{figure}[htb] %
\centering %
\includegraphics[width=#1\linewidth]{#2}%
%\includegraphics[height=#1\linewidth,angle=-90]{#3} %
%\figurebox{#1\linewidth}{}{}[#2]\\
\caption{#3}%
\label{#4}%
\end{figure}}}

\newcommand{\singlefigurerotate}[4]{{\begin{figure}[htb] %
\centering %
\includegraphics[width=#1\linewidth,angle=-90]{#2}%
%\includegraphics[height=#1\linewidth,angle=-90]{#3} %
%\figurebox{#1\linewidth}{}{}[#2]\\
\caption{#3}%
\label{#4}%
\end{figure}}}

\newcommand{\orthogonal}{%
\mathrel{\raisebox{.1em}{%     
\reflectbox{\rotatebox[origin=c]{90}{$\models$}}}}}


%%
%% \BibTeX command to typeset BibTeX logo in the docs
\AtBeginDocument{%
  \providecommand\BibTeX{{%
    \normalfont B\kern-0.5em{\scshape i\kern-0.25em b}\kern-0.8em\TeX}}}

%% Rights management information.  This information is sent to you
%% when you complete the rights form.  These commands have SAMPLE
%% values in them; it is your responsibility as an author to replace
%% the commands and values with those provided to you when you
%% complete the rights form.
\setcopyright{acmcopyright}
\copyrightyear{2020}
\acmYear{2020}
\acmDOI{}

%% These commands are for a PROCEEDINGS abstract or paper.
%\acmConference[Woodstock '18]{Woodstock '18: ACM Symposium on Neural
%  Gaze Detection}{June 03--05, 2018}{Woodstock, NY}
%\acmBooktitle{Woodstock '18: ACM Symposium on Neural Gaze Detection,
%  June 03--05, 2018, Woodstock, NY}
%\acmPrice{15.00}
%\acmISBN{978-1-4503-XXXX-X/18/06}


%%
%% Submission ID.
%% Use this when submitting an article to a sponsored event. You'll
%% receive a unique submission ID from the organizers
%% of the event, and this ID should be used as the parameter to this command.
%%\acmSubmissionID{123-A56-BU3}

%%
%% The majority of ACM publications use numbered citations and
%% references.  The command \citestyle{authoryear} switches to the
%% "author year" style.
%%
%% If you are preparing content for an event
%% sponsored by ACM SIGGRAPH, you must use the "author year" style of
%% citations and references.
%% Uncommenting
%% the next command will enable that style.
%%\citestyle{acmauthoryear}

%%
%% end of the preamble, start of the body of the document source.
\begin{document}

%%%%%%%%%%%---SETME-----%%%%%%%%%%%%%
\title{Entropy Maximization in Sparse Matrix by Vector Multiplication }

\author{Abhishek Jain}
\author{Ismail Bustany}
\author{Henri Fraisse}
\author{Mansmran Benipal}
\author{Dinesh Gaitonde}
\author{Paolo D'Alberto}


%\affiliation{%
%  \institution{1 * -}
%  \streetaddress{2 * -}
%  \city{3 * -}
%  \state{4 * -}
%  \postcode{5 * -}
%}

\renewcommand{\shortauthors}{Jain et al.}

\begin{abstract}
The peak performance of any SpMV accelerator depends primarily on the
available DRAM memory bandwidth and the capability of the accelerator
to effectively use it.  Because SpMV is memory-bound, a more important
metric than peak performance alone is the fraction of bandwidth
utilized, which captures the overall efficiency of the
architecture. GPUs along with some DSA ASIC's and new FPGA
architectures exhibit very high bandwidth: utilizing this much
bandwidth efficiently is difficult for large scale and highly sparse
matrices due to very high random access pattern and workload
imbalance.  We propose a matrix permutation pre-processing step that
aims to maximize the entropy of the distribution of the nonzero
elements.  We conjecture this would be most effective for matrices
with no dense rows or columns and, as in preconditioning, when the
matrix is reused.  Unlike matrix ordering schemes that seek to reduce
fill, we seek a permutation that uniformly distributes the non-zero
elements' distribution, thereby generating a SpMV problem that is
amenable to work load balancing or to speed up sort algorithms.  We
shall show that entropy maximization is an optimization that any
architecture may take advantage although in different ways. Most
importantly, any developer can consider and deploy. We shall present
cases where we can improve performance by 15\% on AMD-based systems.

\end{abstract}

\maketitle

\section{Introduction} 
\label{sec:introduction}

To define the scope of this work, the obvious questions to ask are:
first, what randomization or entropy maximization is in the context of
sparse matrices; second, why would we use it; third, when it does
work.  We shall provide formal definitions in the following
sections. Briefly, we will permute randomly the rows and columns of a
sparse matrix before multiplying it with a dense vector (SpMV) with
the aim of speeding this operation.  Undoubtedly, this scheme requires
some restrictions about the matrix structure, one among them is that
is has no or few dense columns or rows.  In the case, where there are
dense columns or rows, a sparse/dense partitioning scheme should be
used. For the remainder of this manuscript, we shall assume the former
nonzero structure. We use randomization because it is the poor man's
way for preconditioning SpMV in our restricted context, and we do not
mean it in a pejorative sense.


Preconditioning speeds up the convergence rate of an iterative linear
solver by linearly transforming the associated matrix into a form that
affords a faster reduction of the residual error at every iteration.
The cost of this transformation is justified by the runtime reduction
it affords.  Likewise, we foresee randomization playing a similar role
for SpMV in the context of iterative linear solvers and other methods
(e.g in convolutions) where the matrix is reused.

Sparse linear algebra and GraphBLAS kernels are memory bound and there
is a common thread in the scientific computing community to develop
acceleration libraries mostly for multi-core systems. These
predominantly include multi-core processors and GPUs. The goal is
a balanced work distribution and, when applicable, minimal
communication
\cite{DBLP:journals/siamsc/KayaaslanAU18,DBLP:conf/ieeehpcs/PageK18}. When
storage strategy and algorithms must be considered together then GPUs
provide the work horse for abundant thrust in research
\cite{DBLP:journals/topc/AnztCYDFNTTW20}. These works aim at optimal
solutions and strive for a clear and complete
understanding/exploitation of the software-hardware interface; usually
the hardware is composed of symmetric computational
units. Interestingly, the SpMV's space and time complexity, which are
small, may not warrant more performance because we typically end up
utilizing only one-thousandth fraction of the available hardware
capacity.

The peak performance of any SpMV accelerator depends primarily on the
available memory bandwidth (i.e., DRAM such as DDR or HBM) and the
capability of the accelerator to effectively use it.  Because SpMV is
memory-bound, a more important metric than peak performance alone is
the fraction of bandwidth utilized, which captures the overall
efficiency of the architecture. GPU platforms exhibit very high
bandwidth, see the experimental Section \ref{sec:experimentalresults}:
Ellesmere DDR5 224GB/s, Fiji HBM 512GB/s, and Vega 20 HBM 1TB/s.
Although utilizing this much bandwidth efficiently is difficult for
large scale and highly sparse matrices due to very high random access
pattern.  Custom architectures based on FPGA or ASIC devices can
maximize bandwidth utilization by highly customized data-paths and
memory hierarchy designs~\cite{,fowers2014high,grigoras2016cask,zhou2019hitgraph}. Most of the
existing accelerators saturate the relatively low memory bandwidth
available on FPGA platforms (less than 80 GB/s)~\cite{nagar2011sparse,townsend2013reduce,fowers2014high,grigoras2016cask,li2016data,zhou2019hitgraph}.  Modern FPGA platforms have multiple HBM stacks to
provide large memory bandwidth. However, there is no implementation
(currently available) that saturates all of the available DRAM
bandwidth for SpMV kernel on HBM-enabled FPGA platforms. Scalability
of accelerator design remains a major concern, and it is an active area
of research.
%\newline

FPGA platforms used in early works exhibit low peak performance due to
the scarcity of external memory bandwidth~\cite{kestur2012towards,zhuo2005sparse,fowers2014high}. 
For example, Microsoft's implementation of SpMV uses an FPGA platform which only has 2 DDR2-400
memory banks with a resulting bandwidth of 6.4 GB/s~\cite{kestur2012towards}.  The accelerator is running at 100 MHz, it reads
64 Bytes of data every cycle, which corresponds to 5 non-zeros at
every cycle (a non-zero is about 12 Bytes). At best, the peak
performance is 10 double precision operations every cycle at 100 MHz,
which is 1 GFLOPS (only). In 2009, Convey systems Inc. released the
Convey HC-1 FPGA platform.  It has 16 DDR2-677 memories resulting in
overall 80 GB/s memory bandwidth~\cite{nagar2011sparse}. The
accelerator logic runs at 150 MHz.  It consumes 512 Bytes of data
every cycle, which corresponds to around 40 non-zeros every cycle.  At
best, the peak performance is 80 double precision operations every
cycle at 150 MHz, which is 12 GFLOPS.
%\newline

One of the key building blocks for custom architecture solutions is a
multi-ported buffer used to storing vector entries~\cite{fowers2014high}.  During execution,
multiple column indices are used as addresses to read corresponding
vector entries; we shall provide more details in Section
\ref{sec:notations}.  Designing a buffer with a very large number of
read ports is challenging.  One solution is {\em banking} as a
mechanism to store partitioned vector entries.  Although banking could
allow very high throughput indexing unless the same entry is required
multiple times and its reads are purely sequential causing loss of
bandwidth.  For example, hashing techniques and data duplication are
possible solutions for this problem. However, another issue arises:
When we distribute SpMV computations across $p$-nodes, some of the
nodes, say $k$, finish later than the rest because of unbalanced work
loads (ie. number of nonzero element) in row/column major
traversal. This is a common phenomena for matrices where few rows or
columns are dense. These $k$ nodes are referred to as {\em laggard
  nodes}.  By applying random permutation of columns/rows, we are
attempting to balance the loads across all $p$ workers so that there
is are no laggards. From this hardware vantage point, randomization or
maximizing the entropy of the non-zero element distribution is an
optimization transform and provides a clear context for our work.
%\newline

Clearly, optimally accelerating SpMV is a hard many-parameters
optimization problem dependent on the choice of algorithm, data
structures, and dedicated hardware (CPU, GPUs, FPGA's, Custom ASIC's).
Rather, our goal is to provide a tool, we may say a naive tool, to
help understand how the structure of the matrix may impact the HW-SW
solution.  For the readers in the field of algorithms, SpMV can be
mapped into a sorting algorithm. For example, finding elements
$x_{i,j}$ and $x_{i,k>j}$ in a sparse matrix requires to find row $i$
and then columns $j$ and $k$.  Sorting is a method to find if an
element is in a list with no prior or limited knowledge of its
contents.  Sorting can be used to prepare the matrix and to find
elements in between sparse matrices and sparse vectors. In custom
architectures, sorting networks are used to route matrix and vector
elements to functional units. In a sense, If one is stuck with a
sorting algorithm and a poor distribution, randomization may alter the
distribution and throttle performance. Interestingly, the best sorting
algorithm is a function of the distribution of the elements
\cite{LiGP2004,HuangGl2009}.

We organize this work as follows: In Section \ref{sec:notations}, we
define the matrix by vector operation; in Section
\ref{sec:randomization}, we define what we mean by randomization (or
entropy maximization). We use randomization to create a uniform
distribution in Section \ref{sec:uniform} and measure uniformity
%by nothing else than
by entropy in Section \ref{sec:entropy}. We present how we drive our
experiments to show the effects of randomization in Section
\ref{sec:measuring}. In the last sections, we present a summary of the
results: we present our task work loads for the given benchmarks in Section
\ref{sec:workload}, and the complete set of measures for an AMD CPU
and GPUs systems in Section \ref{sec:experimentalresults} .

\section{Basic Notations}
\label{sec:notations}
Let us start by describing the basic notations so we can clear the
obvious (or not).  A Sparse-matrix by vector multiplication {\em SpMV}
on an (semi) ring based on the operations $(+,*)$ is defined as
$\Vc{y} = \M \Vc{x}$ so that $y_i = \sum_j M_{i,j}*y_j$ where $M_{i,j}
\eq 0$ are not represented nor stored. Most of the experimental
results in Section \ref{sec:experimentalresults} are based on the
classic addition (+) and multiplication (*) in floating point
precision using 64 bits (i.e., double floating point precision) albeit
are extensible to other semi-rings.  For instance, it is well known
that SpMV defined on the semi-ring (min,+) is a kernel in computing an
all-pairs shortest paths starting with a graph adjacency matrix, and
in using a Boolean algebra we can check if two nodes are connected,
which is slightly simpler.
%\newline

We identify a sparse matrix $\M$ of size $M\times N$ as having
$O(M+N)$ non-zero elements, number of non zero {\em nnz}. Thus the
complexity of $\M \Vc{x}$ is $O(M+N) \approx 2nnz$. Of course, the
definition of sparsity may vary. We represent the matrix $\M$ by using
the coordinate list {\em COO} or and the compressed sparse row {\em
  CSR}\footnote{a.k.a. Compressed row storage {\rm CRS}.}  formats. The
COO represents the non-zero of a matrix by a triplet $(i,j,v)$; very
often there are three identical-in-size vectors for the ROW, COLUMN,
and VALUE. The COO format takes $3\times nnz$ space and two
consecutive elements in the value array are not bound to be neither in
the same row nor column. In fact, we know only that $VALUE[i] =
M_{ROW[i],COLUMN[i]}$.

The CSR format stores elements in the same row and with increasing
column values consecutively. There are three arrays V, COL, and
ROW. The ROW is sorted in increasing order.  Its size is $M$, and
$ROW[i]$ is an index in V and COL describing where $i$-th row starts
(i.e., if row $i$ exists).  Accordingly, $M_{i,*}$ is stored in
$V[ROW[i]:ROW[i+1]]$. The column indices are stored at
$COL[ROW[i]:ROW[i+1]]$ and sorted increasingly. The CSR format takes
$2\times nnz + M$ space and a row vector of the matrix can be found in
$O(1)$.
%\newline

The computation  $y_i = \sum_j M_{i,j}*x_j$ is a sequence of scalar
products and, using the CSR format, is computed as follows:

\[ Index = ROW[i]:ROW[i+1] \]
\[
y_i =  \sum_{\ell\in Index} V[\ell] * x_{COL[\ell]}  
\]
%\newline

The matrix row is contiguous (in memory) and rows are stored in
increasing order. However, the access of the dense vector $\Vc{x}$ has
no particular pattern, well increasing.
%\newline

The COO format can be endowed with certain properties. For example, we
can sort the array by row and add row information to achieve the same
properties of CSR. In contrast, transposing a "sorted" COO matrix
simply entails swapping of the arrays ROW and COL. Think about matrix
multiply.  Each scalar product achieves peak performance if the reads
of the vector $\Vc{x}$ are streamlined as much as possible and so the
reads of the vector $V$. If we have multiple cores, each could compute
a subset of the $y_i$ and a clean data load balancing can go a long
way. If we have few functional units, we would like to have a constant
stream of independent $*$ and $+$ operations but with data already in
registers. That is, data pre-fetch will go a long way especially for
$x_{COL[i]}$, which may have an irregular pattern.


\section{Randomization or Entropy Maximization}
\label{sec:randomization}
We define {\em Randomization} as row or column permutation transform
of the matrix $\M$ (thus a permutation of $\Vc{y}$ and $\Vc{x}$), and
we choose these by a pseudo-random process. The obvious question to as
is why should we seek randomization (or entropy maximizing) transform?
The sparsity of a given matrix $\M$ has a non-zero element
distribution induced by the nature of the original problem or by some
imposed ordering on the respective nodes of its associated graph.
This distribution may be computationally incompatible with the chosen
algorithm or architecture. For instance, it can induce some load
imbalance in the computation.  We could break this load imbalance by
seeking to maximize entropy for this distribution. Our conjecture is
that would favor the average case performance rather than the worse
case when operating on the "max-entropy transformed" matrix.

For linear system solvers, if we know the matrix $\M$, and we know the
architecture, preconditioning (when affordable) is a better solution.
Well, it is. If we run experiments long enough, we choose the best
permutation(s) for the architecture, permute $\M$, and go on testing
the next.  On one end, preconditioning exerts a full understanding of
both the matrix (the problem) and how the final solution will be
computed (architecture). This is the culminating point of knowing, and
we must strive to it. On the other end, the simplicity of a random
permutation requires no information about the matrix, the vector, and
the architecture. Such a simplicity can be exploited directly in
Hardware. We are after an understanding when randomization is just
enough: We seek to let the hardware do its best with the least effort,
or at least with the appearance to be effortless.
%Also we shall show there are different flavors of randomization.


Interestingly, this work stems from a sincere surprise about
randomization efficacy and its application on custom SpMV. Here, we
wish to study this problem systematically so that to help future
hardware designs. Intuitively, if we can achieve a uniform
distribution of the rows of matrix $\M$ we can have provable
expectation of its load balancing across multiple cores. If we have a
uniform distribution of accesses on $\Vc{x}$ we could exploit column
load balancing and exploit better sorting algorithms: In practice, the
reading of $\Vc{x}_{COL[i]}$ can be reduced to a sorting, and there we know
that different sparsity  may require different algorithms. This may be 
lot to unpack but it translates to a better performance of the
sequential algorithm without changing the algorithm or to improved bandwidth
utilization.


We will show that (different) randomness affects architectures and
algorithms differently, making randomization a suitable optimization
transform especially when the application and hardware are at odds:
Hardware (unless programmable) is difficult to change and the matrix
sparsity is simple to change. We want to show that there is a
randomness hierarchy that we can distinguish as global and
local. There are simple-to-find cases where the sparsity breaks
randomness optimization.  For instance, matrices with dense rows or
columns are better partitioned into sparse and dense components and
operated on separately.
%We want to show that this study uses common tool, open
%software tools and sometimes naive experiments; however, we can infer
%properties applicable to proprietary and custom solutions.
%Hi Paolo, above two sentences need further clarification
\newpage 
\Doublefigure{.49}{../python/PNG/OPF_3754_mtx_regular}{../python/PNG/lp_osa_07_mtx_regular}{Left: OPF 3754. Right: LP OSA 07. These are histograms where we represent normalized buckets and counts}{fig:one}

\section{Entropy}
\label{sec:entropy}
Patterns in sparse matrices are often visually pleasing, see Figure
\ref{fig:one} where we present the height histogram, the width
histograms and a two-dimensional histogram as heat map. We will let
someone else using AI picture classification. Intuitively, we would
like to express a measure of uniform distribution and here we apply
the basics: {\em Entropy}. Given an histogram $i\in[0,M-1]$ $h_i \in
\N$, we define $S =\sum_{i=0}^{M-1}h_i$ and thus we have a probability
distribution function $p_i = \frac{h_i}{S}$. The {\em information} of
bin $i$ is defined as $I(i) = -\log_2 p_i$. If we say that the
stochastic variable $X$ has PDF $p_i$ than the entropy of $X$ is
defined as.

\begin{equation}
  \label{eq:entropy}
  H(x) = -\sum_{i=0}^{M-1} p_i\log_2p_i = \sum_{i=0}^{M-1}p_i I(i) =
  E[I_x]
\end{equation}
The maximum entropy is when $\forall i, p_i = p = \frac{1}{M}$; that
is, we are observing a uniform distributed event. There is no
conceptual difference when the PDF represents a two dimensional
distribution. Thus our randomization should aim at higher entropy
numbers. The entropy for matrix LP OSA 07 is 8.41 and for OPF 3754 is
8.39. We use the entropy specified in the Scipy stats module.  A
single number is concise and satisfying. If you are pondering why they
are so close contrary to their sparsity we discuss this next.


\section{Uniform distribution}
\label{sec:uniform}
We know that we should {\bf not} compare the entropy numbers of two
matrices because entropy does not use any information about the order
of the buckets only their probabilities. By construction, the matrices
are quite different in sparsity and in shapes, however their entropy
numbers are very close.  Two matrices with the same number of
non-zeros, spaced well enough in the proper number of bin, will have
the same entropy. To appreciate their different sparsity, we should
compare their entropy distributions by Jensen-Shannon measure (which
is a symmetric measure, please do not use Kullback-Leibler KL
divergence) \cite{dalberto2012nonparametric}. Or we could use a representation of a hierarchical
2d-entropy, see Figure \ref{fig:two}, where the entropy is split into
2x2, 4x4 and 8x8 (or fewer if the distribution is not square). We have
a hierarchical entropy heat maps.

\doublefigure{.40}{../python/ENTOPF3754/2d-regular}{../python/ENTlposa07/2d-regular}{Hierarchical 2D entropy for OPF 3754 (left) and LP OSA 07 (right). }{fig:two}

We can see that a granular entropy summarizes better the nature of the
matrix because it keep some spatial information. In this work, the
entropy vector is used mostly for visualization purpose more than for
comparison purpose. Of course, we can appreciate how the matrix LP OSA
07 has a few very heavy rows and they are clustered. This matrix will
help us showing how randomization need some tips. Now we apply row and
column random permutation once by row and one by column: Figure
\ref{fig:three}: OPF has now entropy 11.27 and LP 9.26. The numerical
difference is significant. The good news is that for entropy, being an
expectation, we can use simple techniques like bootstrap to show that
the difference is significant or we have shown that Jensen-Shannon can
be used and a significance level is available. What we like to see is
the the hierarchical entropy heat map is becoming {\em more} uniform
for at least one of the matrix.

\doublefigure{.40}{../python/ENTOPF3754/2d-row-column-shuffle}{../python/ENTlposa07/2d-row-column-shuffle}{Hierarchical 2D entropy after row and column random permutation for OPF 3754 (left) and LP OSA 07 (right). }{fig:three}

In practice, permutations need some help especially for relatively
large matrices. As you can see, the permutation affects locally the
matrix. Of course, it depends on the implementation of the random
permutation, we use {\em numpy} for this. It is reasonable that a
slightly modified version of the original is still a random selection
and unfortunately they seem too likely in practice. We need to
compensate or help the randomization. If we are able to identify the
row and column that divide high and low density, we could use them as
pivot for a shuffle like in a quick-sort algorithm. We could apply a
sorting algorithm but its complexity will the same of SpMV. We use a
gradients operations to choose the element with maximum steepness,
Figure \ref{fig:four} and \ref{fig:five}

LP achieves entropy 8.67 and 9.58 and OPF achieves 10.47 and 11.40.

\doublefigure{.40}{../python/ENTOPF3754/2d-H-shuffle}{../python/ENTlposa07/2d-H-shuffle}{Hierarchical 2D entropy after height gradient based shuffle and row random permutation for OPF 3754 (left) and LP OSA 07 (right). }{fig:four}

\doublefigure{.40}{../python/ENTOPF3754/2d-W-shuffle}{../python/ENTlposa07/2d-W-shuffle}{Hierarchical 2D entropy after height and width gradient shuffle and row and column random permutation for OPF 3754 (left) and LP OSA 07 (right). }{fig:five}

If the goal is to achieve a uniformly sparse matrix, it seems that we
have the tools to compute and to measure such a sparsity. We admit
that we do not try to find the best permutation. But our real goal is
to create a work bench where randomization can be tested on different
architectures and different algorithms. A randomization with a
measurable uniform distribution is preferable than just random. We are
interested to find out when random is enough or not enough. Also,
consider that to achieve a uniform distribution, we do not need a
random transformation and any permutation balancing the number of
non-zero is possible, but for now not looked for.

\section{Measuring the randomization effects}
\label{sec:measuring}

Whether or not this ever applied to the reader, when we have timed
algorithms (i.e., measure execution time), we came to expect
variation.  The introduction of randomization may hide behind the ever
present variance, after all these are algorithms on {\em small} inputs
and small error can be comparable to the overall execution time. Here,
we must address this concern even before describing the experiments.

First, we execute every algorithm between 1000 and 5000 times. The
time of each experiment is in the seconds, providing a granularity for
which we are confident the measuring time error is under
control. Thus, for each experiment we provide an average execution
time: we measure the time and we divide by the number of trials. Cold
starts, the first iteration, are still accounted. To make the measure
portable across platform we present GFLOPS, that is, Giga ($10^{12}$)
floating operations per second: $2*nnz$ divided by the average time in
seconds.

Then we repeat the same experiment 32 times. Permutations in {\em
  numpy} Python uses a seed that is time sensitive: thus every
experiment is independent from the previous. The number 32 is an old
statistic trick and it is a minimum number of independent trials to
approximate a normal distribution. In practice, they are not but the
number is sufficient for most of the cases and it is an excellent
starting point.

A short hand legend: {\bf Reg} is the matrix without any permutation
and thus is the regular; {\bf R} stands for random Row permutation;
{\bf G-R} stands for gradient-based row shuffle and random row
permutation; {\bf G-C} stands for gradient-based column shuffle and
random column permutation; {\bf R-C} stands for random row and column
permutation.  This legend is used in the pictures to be concise, in
the tables in the following sections, we use a verbose description. We
shall clarify the gradient based approach in the experimental results
section \ref{sec:experimentalresults}. Intuitively, we help the random
permutation by a quick targeting of high and low volume of the
histogram (and thus the matrix).


In Figure \ref{fig:five1}, We show CPU performance using COO and CSR
SpMV algorithms for the matrix OPF 3754. The figure is an histogram:
The $x$ is GFLOPS and the $y$ label is the number of counts. Thus we
show what is the performance distribution of an algorithm.  We can see
that the CSR algorithms are consistent and the Regular (i.e., the
original) has always the best performance. Also the variance of the
computation time is small and the shape is approximately Gaussian.
Different story for the COO, the permutations introduce long tails,
thus $2\times$ performance advantage.
\doublefigure{.45}{../python/PNG2/OPF_3754_mtx_CPU_COO.png}{../python/PNG2/OPF_3754_mtx_CPU_CSR.png}{CPU
  COO (left) and CPU CSR (left) for OPF 3754}{fig:five1}

\singlefigure{.45}{../python/PNG2/OPF_3754_mtx_CPU_PAR.png}{ Parallel CPU CSR for OPF 3754}{fig:seven}
If we take the original matrix and split into parts having the same
number of rows, and execute them in parallel using different cores, we
can see in Figure \ref{fig:seven} that randomization is quite useful.


In Figure \ref{fig:six}, \ref{fig:six1} and \ref{fig:six2},
randomization is harmful to the GPU implementation. The OPF 375 matrix
is mostly diagonal, thus the vector $\Vc{x}$ is read in close
quarters, randomization breaks it.  If the load balance is fixed
(i.e., by dividing the matrix by row and in equal row), randomization
is beneficial.

\doublefigure{.45}{../python/PNG2/OPF_3754_mtx_GPU_64_COO.png}{../python/PNG2/OPF_3754_mtx_GPU_64_CSR.png}{Vega 20, GPU 64bits COO (left) and GPU CSR (right) for OPF 3754}{fig:six}
\doublefigure{.45}{../python/ELLEPNG/OPF_3754_mtx_GPU_64_COO.png}{../python/ELLEPNG/OPF_3754_mtx_GPU_64_CSR.png}{Ellesmere, GPU 64bits COO (left) and GPU CSR (right) for OPF 3754}{fig:six1}
\doublefigure{.45}{../python/FIJIPNG/OPF_3754_mtx_GPU_64_COO.png}{../python/FIJIPNG/OPF_3754_mtx_GPU_64_CSR.png}{Fiji, GPU 64bits COO (left) and GPU CSR (right) for OPF 3754}{fig:six2}

\newpage

For matrix LP OSA 07, randomization helps clearly only for CPU CSR as
we show in Figure \ref{fig:eight}. In Figure \ref{fig:eight1},
\ref{fig:eight2}, and \ref{fig:eight3}, we can see that randomization
is harmful but for one GPU, we can show that a single exception is
possible (40\% improvement).

\singlefigure{.45}{../python/PNG2/lp_osa_07_mtx_CPU_CSR.png}{   CPU CSR  for LP OSA 07}{fig:eight}

\doublefigure{.45}{../python/PNG2/lp_osa_07_mtx_GPU_64_COO.png}{../python/PNG2/lp_osa_07_mtx_GPU_64_CSR.png}{Vega 20, GPU 64bits COO (left) and GPU CSR (right) for OPF 3754}{fig:eight1}
\doublefigure{.45}{../python/ELLEPNG/lp_osa_07_mtx_GPU_64_COO.png}{../python/ELLEPNG/lp_osa_07_mtx_GPU_64_CSR.png}{Ellesmere, GPU 64bits COO (left) and GPU CSR (right) for OPF 3754}{fig:eight2}
\doublefigure{.45}{../python/FIJIPNG/lp_osa_07_mtx_GPU_64_COO.png}{../python/FIJIPNG/lp_osa_07_mtx_GPU_64_CSR.png}{Fiji, GPU 64bits COO (left) and GPU CSR (right) for OPF 3754}{fig:eight3}

\newpage
An example, the matrix MULT DCOP 01, is where randomization is useful
for the CPU, GPU, and the parallel version Figure \ref{fig:9},
\ref{fig:10} - \ref{fig:11} and the gains can be up to
10-15\%. Consider, we can achieve these improvements without any
insights to the architecture, the algorithms and their relationships.


\doublefigure{.45}{../python/PNG2/mult_dcop_01_mtx_CPU_COO.png}{../python/PNG2/mult_dcop_01_mtx_CPU_CSR.png}{CPU COO (left) and CPU CSR (right) for MULT DCOP 01}{fig:9}
\doublefigure{.45}{../python/PNG2/mult_dcop_01_mtx_GPU_64_COO.png}{../python/PNG2/mult_dcop_01_mtx_GPU_64_CSR.png}{Vega 20, GPU 64bits COO (left) and GPU CSR (right) for MULT DCOP 01}{fig:10}
\doublefigure{.45}{../python/ELLEPNG/mult_dcop_01_mtx_GPU_64_COO.png}{../python/ELLEPNG/mult_dcop_01_mtx_GPU_64_CSR.png}{Ellesmere,  GPU 64bits COO (left) and GPU CSR (right) for MULT DCOP 01}{fig:101}
\doublefigure{.45}{../python/FIJIPNG/mult_dcop_01_mtx_GPU_64_COO.png}{../python/FIJIPNG/mult_dcop_01_mtx_GPU_64_CSR.png}{Fiji, GPU 64bits COO (left) and GPU CSR (right) for MULT DCOP 01}{fig:102}
\singlefigure{.45}{../python/PNG2/mult_dcop_01_mtx_CPU_PAR.png}{ Parallel
  CPU CSR  for MULT DCOP 01 }{fig:11}

What does it mean when randomization does not work? The matrices we
use in this work are not chosen randomly (pun not intended), they are
the matrices that are difficult to handle in our custom SpMV engines
using a combination of sorting networks and systolic arrays. If
randomization does not work in our simplified work bench, will not
work in our specialized architecture because the reorganization of the
matrix or the input and output vector does not have the necessary
parallelism, data locality, and data streaming. We need to do
something else. In this case disrupting the memory pattern is not
sufficient. Thus, if we cannot beat the pattern, we must exploit it,
well not in this work.

\section{Workloads}
\label{sec:workload}

In the previous sections, we defined what we mean for randomization
and we present our tools of tricks for the measure of the effects of
randomization. Here we describe the work loads, the applications, we
use to test the effects of the randomization.

\subsection{Python COO and CSR algorithms}
\label{sec:pythoncoocsr}
The simplicity to compute the SpMV by the code $z = A*b$ in Python is
very rewarding. By change of the matrix storage format, $A =
A.tocsr(); z = A*b$, we have a different algorithm. The performance
exploitation is moved to the lower level.  The CSR implementation is
often two times faster but there are edge cases where the COO and COO
with randomization can go beyond and be surprisingly better: MUL DCOP
03 is an example where COO can do well.

Intuitively, Randomization can affect the performance because the
basic implementation is a sorting algorithm and it is a fixed
algorithm. There are many sorting algorithms and each can be optimal
for a different initial distribution. If we knew what is the sorting
algorithm we could tailor the input distribution. Here we just play
with it.

In Section \ref{sec:experimentalresults}, we present all the results
for CPU and GPUS. Keep in mind that these problems are hard, in the
sense they do not have fancy performance sheets (these architectures
can achieve Tera FLOPs sustained performance for dense computations).
If we go through diligently, we can see that there is a 15x
performance difference between the single thread CPU and Vega 20 GPU
(i.e, 3 vs 40 GFLOPS).

\subsection{Parallel CSR using up to 16 cores}
\label{sec:parcpu}
Python provides the concept of Pool to exploit a naive parallel
computation. We notice that work given to a Pool is split accordingly
to the number of elements to separate HW cores. We also noticed that
the work load move from a core to another, thus not ideal. Also we
notice that Pool introduce a noticeable overhead: a Pool of 1, never
achieves the performance of the single thread $z = A*b$. Using Pool
allows us to investigate how a naive row partitioning without counting
can scale up with number of cores. We tested by splitting the rows to
1--16 cores evenly (one thread per core) and we present the
performance for only the best configuration. The randomization goal is to
distribute the work uniformly: a balanced work distribution avoid the
unfortunate case where a single core does all the work. We are pleased
by the simplicity of the benchmark and we know we can do better.



\subsection{GPU COO and CSR algorithms}
\label{sec:gpucoocsr}
In this work, we use AMD GPUs and {\em rocSPARSE} is their current
software. The software has a few glitches but overall can be used for
different generation of AMD GPUs. We use the COO and CSR algorithms
and we provide performance measure for double precision only. The
ideas of using different GPUs: it is important to verify that the
randomization can be applied independently of the HW. We are not here
to compare performance across GPUs and CPUs. Often the limitation is
the software, how the software can exploit the hardware or how the
software will make easy to use a specific GPU. For example, the Fiji
architecture is clearly superior to the Ellesmere, however the latter
have better support and the system overall is more stable and user
friendly.

The performance of the CSR algorithm is about two times faster than the
COO. Most of the algorithms count the number of sparse elements in a
row and thus they can decide the work load partition
accordingly. Counting give you an edge but without changing the order
of the computation there could be cases where the work load is not
balanced and a little randomization could help and it does.

%\subsection{ FPGA ? (not necessary)}
\subsection{Randomization sometimes works}

For the majority of the cases we investigated and reported in the
following sections, Randomization does not work. However, there are
cases where randomization does work and does work for different
algorithms and architectures. If you are in the business of
preconditioning, permutations are pretty cheap. If you can find a good
one just consider like a preconditioning matrix, which it is. 

This shows also that HW has to be more conscious, well the HW designer
should, and accept that there are options at software level, at matrix
level and beyond. 


%\section{Call for a different strategy}
%\label{sec:strategy}
%We want to find out randomization techniques that are suitable for
%custom hardware but also what are the most common and simple
%heuristics that can justified for any hardware.


\section{Experimental Results}
\label{sec:experimentalresults}
The main hardware setup is a AMD Threadripper with 16 cores. We have
three Radeon GPUs: Vega 20 7nm, Pro 2xFiji, and Pro 2xEllesmere.

Vega 20 can deliver 3.5TFLOPS in double precision and it has 1TB/s HBM
memory. Each Fiji provides 0.5 TFLOPS in double precision and has
512GB/s HBM, the card has two chips.  The Ellesmere provides 0.3TFLOPS
in double precision and has 224GB/s DDR5, the card has two chips. In
the performance plots presented earlier and in the following, you will
notice that the performance gap between these GPUs is not so
marked. We can safely state that $vega \sim 2\times Fiji$ and $Fiji \sim
2\times ellesmere$



There are 4 basic randomization formats:
\begin{itemize}
  \item {\bf Random Row Permutation}, we take the original matrix and
    permute the rows.
  \item {\bf Random Row and Column Permutation}, we take the original
    matrix and permute the rows and the columns.
  \item {\bf Gradient based row permutation}, we compute the row
    histogram and we compute the gradient: $h_{i+1} - h_i$. We find a
    single point where the gradient is maximum, this is the pivot for
    a shuffle like a magician would shuffle a deck of cards.  Then we
    permute the two parts randomly.
  \item {\bf Gradient based row and column permutation}, As above but
    also for the columns.
\end{itemize}

For large matrices (large number of columns and rows) a permutation
tends to be a close variation of the original, still a random
permutation. The gradient allows us to describe two area of the
original matrix where there is a clear and de-marked density
variation: for example, there are two uniform distributed sub matrices
but one denser than the other. A shuffle redistribute every other
sample/card to different parts and these can be permuted locally.


    
We report in the following the performance results, we introduce a
{\bf *} following the best performance. This is tedious to read and,
we assure, to write. The code and the results are available as
software repository. \section{Vega VII}
\label{sec:vega}
\small
\begin{verbatim}
mult_dcop_03.mtx
 Regular
                          CPU COO    min  0.728 max  0.880 mean  0.757
                          CPU CSR    min  1.563 max  1.581 mean  1.577
                          GPU 32 COO min  0.000 max  0.000 mean  0.000
                                 CSR min  0.000 max  0.000 mean  0.000
                          GPU 64 COO min  8.540 max  8.670 mean  8.619
                                 CSR min 18.320 max 18.930 mean 18.620
                          CPU PAR    min  1.170 max  1.269 mean  1.226
                          H          min  9.689 max  9.689 mean  9.689
 Row-Premute
                          CPU COO    min  0.710 max  0.845 mean  0.724
                          CPU CSR    min  1.549 max  1.597 mean  1.589
                          GPU 32 COO min  0.000 max  0.000 mean  0.000
                                 CSR min  0.000 max  0.000 mean  0.000
                          GPU 64 COO min  8.360 max  8.540 mean  8.442
                                 CSR min 16.260 max 16.780 mean 16.551
                          CPU PAR    min  1.205 max  1.319 mean  1.263
                          H          min 10.737 max 10.742 mean 10.740
 Row-Gradient
                          CPU COO    min  0.706 max  1.603 mean  0.806
                          CPU CSR    min  1.493 max  1.534 mean  1.528
                          GPU 32 COO min  0.000 max  0.000 mean  0.000
                                 CSR min  0.000 max  0.000 mean  0.000
                          GPU 64 COO min  8.430 max  8.610 mean  8.527
                                 CSR min 17.070 max 18.970 mean 18.115
                          CPU PAR    min  1.331 max  1.695 mean  1.513
                          H          min 10.576 max 10.585 mean 10.580
 Column-Gradient
                          CPU COO    min  0.694 max  1.632 mean  0.797
                          CPU CSR    min  1.491 max  1.534 mean  1.529
                          GPU 32 COO min  0.000 max  0.000 mean  0.000
                                 CSR min  0.000 max  0.000 mean  0.000
                          GPU 64 COO min  8.350 max  8.520 mean  8.429
                                 CSR min 15.970 max 18.180 mean 17.124
                          CPU PAR    min  1.321 max  1.728 mean  1.514
                          H          min 10.826 max 10.840 mean 10.833
 Row-Column-Permute
                          CPU COO    min  0.688 max  0.757 mean  0.696
                          CPU CSR    min  1.490 max  1.595 mean  1.584
                          GPU 32 COO min  0.000 max  0.000 mean  0.000
                                 CSR min  0.000 max  0.000 mean  0.000
                          GPU 64 COO min  8.380 max  8.500 mean  8.445
                                 CSR min 16.230 max 16.780 mean 16.513
                          CPU PAR    min  1.192 max  1.274 mean  1.237
                          H          min 10.737 max 10.742 mean 10.740
mult_dcop_01.mtx
 Regular
                          CPU COO    min  0.710 max  1.453 mean  0.761
                          CPU CSR    min  1.561 max  1.581 mean  1.578
                          GPU 32 COO min  0.000 max  0.000 mean  0.000
                                 CSR min  0.000 max  0.000 mean  0.000
                          GPU 64 COO min  8.520 max  8.670 mean  8.597
                                 CSR min 18.320 max 18.870 mean 18.636
                          CPU PAR    min  1.163 max  1.246 mean  1.212
                          H          min  9.689 max  9.689 mean  9.689
 Row-Premute
                          CPU COO    min  0.699 max  1.305 mean  0.745
                          CPU CSR    min  1.585 max  1.597 mean  1.590
                          GPU 32 COO min  0.000 max  0.000 mean  0.000
                                 CSR min  0.000 max  0.000 mean  0.000
                          GPU 64 COO min  8.360 max  8.520 mean  8.446
                                 CSR min 16.260 max 16.780 mean 16.528
                          CPU PAR    min  1.192 max  1.298 mean  1.242
                          H          min 10.738 max 10.742 mean 10.740
 Row-Gradient
                          CPU COO    min  0.709 max  1.656 mean  0.819
                          CPU CSR    min  1.527 max  1.535 mean  1.530
                          GPU 32 COO min  0.000 max  0.000 mean  0.000
                                 CSR min  0.000 max  0.000 mean  0.000
                          GPU 64 COO min  8.450 max  8.680 mean  8.527
                                 CSR min 16.520 max 19.480 mean 17.984
                          CPU PAR    min  1.280 max  1.704 mean  1.485
                          H          min 10.572 max 10.585 mean 10.581
 Column-Gradient
                          CPU COO    min  0.698 max  1.042 mean  0.737
                          CPU CSR    min  1.458 max  1.536 mean  1.528
                          GPU 32 COO min  0.000 max  0.000 mean  0.000
                                 CSR min  0.000 max  0.000 mean  0.000
                          GPU 64 COO min  8.340 max  8.600 mean  8.443
                                 CSR min 16.360 max 18.450 mean 17.247
                          CPU PAR    min  1.307 max  1.712 mean  1.494
                          H          min 10.823 max 10.841 mean 10.835
 Row-Column-Permute
                          CPU COO    min  0.683 max  1.247 mean  0.749
                          CPU CSR    min  1.583 max  1.595 mean  1.590
                          GPU 32 COO min  0.000 max  0.000 mean  0.000
                                 CSR min  0.000 max  0.000 mean  0.000
                          GPU 64 COO min  8.370 max  8.500 mean  8.435
                                 CSR min 16.250 max 16.780 mean 16.518
                          CPU PAR    min  1.206 max  1.291 mean  1.243
                          H          min 10.738 max 10.742 mean 10.740
mult_dcop_02.mtx
 Regular
                          CPU COO    min  1.615 max  1.677 mean  1.652
                          CPU CSR    min  1.539 max  1.579 mean  1.575
                          GPU 32 COO min  0.000 max  0.000 mean  0.000
                                 CSR min  0.000 max  0.000 mean  0.000
                          GPU 64 COO min  8.530 max  8.700 mean  8.614
                                 CSR min 18.290 max 18.890 mean 18.597
                          CPU PAR    min  1.120 max  1.248 mean  1.211
                          H          min  9.689 max  9.689 mean  9.689
 Row-Premute
                          CPU COO    min  0.684 max  0.780 mean  0.705
                          CPU CSR    min  1.558 max  1.596 mean  1.588
                          GPU 32 COO min  0.000 max  0.000 mean  0.000
                                 CSR min  0.000 max  0.000 mean  0.000
                          GPU 64 COO min  8.360 max  8.490 mean  8.433
                                 CSR min 16.240 max 16.750 mean 16.552
                          CPU PAR    min  1.182 max  1.277 mean  1.242
                          H          min 10.737 max 10.742 mean 10.740
 Row-Gradient
                          CPU COO    min  0.704 max  1.373 mean  0.790
                          CPU CSR    min  1.518 max  1.535 mean  1.529
                          GPU 32 COO min  0.000 max  0.000 mean  0.000
                                 CSR min  0.000 max  0.000 mean  0.000
                          GPU 64 COO min  8.420 max  8.590 mean  8.517
                                 CSR min 16.680 max 19.550 mean 17.907
                          CPU PAR    min  1.328 max  1.713 mean  1.484
                          H          min 10.572 max 10.585 mean 10.581
 Column-Gradient
                          CPU COO    min  0.697 max  1.460 mean  0.742
                          CPU CSR    min  1.517 max  1.534 mean  1.527
                          GPU 32 COO min  0.000 max  0.000 mean  0.000
                                 CSR min  0.000 max  0.000 mean  0.000
                          GPU 64 COO min  8.330 max  8.490 mean  8.420
                                 CSR min 16.020 max 18.390 mean 17.303
                          CPU PAR    min  1.321 max  1.709 mean  1.557
                          H          min 10.823 max 10.843 mean 10.835
 Row-Column-Permute
                          CPU COO    min  0.691 max  0.746 mean  0.698
                          CPU CSR    min  1.568 max  1.595 mean  1.587
                          GPU 32 COO min  0.000 max  0.000 mean  0.000
                                 CSR min  0.000 max  0.000 mean  0.000
                          GPU 64 COO min  8.350 max  8.500 mean  8.436
                                 CSR min 16.250 max 16.780 mean 16.517
                          CPU PAR    min  1.187 max  1.280 mean  1.228
                          H          min 10.739 max 10.743 mean 10.740
lp_fit2d.mtx
 Regular
                          CPU COO    min  0.774 max  0.804 mean  0.793
                          CPU CSR    min  2.538 max  2.550 mean  2.547
                          GPU 32 COO min  0.000 max  0.000 mean  0.000
                                 CSR min  0.000 max  0.000 mean  0.000
                          GPU 64 COO min  7.060 max  7.170 mean  7.101
                                 CSR min 15.650 max 18.700 mean 18.031
                          CPU PAR    min  1.537 max  1.645 mean  1.590
                          H          min 11.109 max 11.109 mean 11.109
 Row-Premute
                          CPU COO    min  0.740 max  0.776 mean  0.746
                          CPU CSR    min  3.302 max  3.328 mean  3.317
                          GPU 32 COO min  0.000 max  0.000 mean  0.000
                                 CSR min  0.000 max  0.000 mean  0.000
                          GPU 64 COO min  7.040 max  7.180 mean  7.098
                                 CSR min 15.690 max 18.580 mean 16.732
                          CPU PAR    min  1.327 max  1.482 mean  1.422
                          H          min 11.098 max 11.105 mean 11.101
 Row-Gradient
                          CPU COO    min  0.739 max  2.092 mean  1.091
                          CPU CSR    min  2.539 max  2.546 mean  2.543
                          GPU 32 COO min  0.000 max  0.000 mean  0.000
                                 CSR min  0.000 max  0.000 mean  0.000
                          GPU 64 COO min  7.040 max  7.150 mean  7.100
                                 CSR min 15.520 max 18.560 mean 17.547
                          CPU PAR    min  1.401 max  1.661 mean  1.525
                          H          min 11.109 max 11.109 mean 11.109
 Column-Gradient
                          CPU COO    min  0.726 max  2.065 mean  1.011
                          CPU CSR    min  2.539 max  2.550 mean  2.546
                          GPU 32 COO min  0.000 max  0.000 mean  0.000
                                 CSR min  0.000 max  0.000 mean  0.000
                          GPU 64 COO min  6.800 max  7.140 mean  7.080
                                 CSR min 15.480 max 18.560 mean 16.866
                          CPU PAR    min  1.391 max  1.737 mean  1.563
                          H          min 11.329 max 11.333 mean 11.331
 Row-Column-Permute
                          CPU COO    min  0.746 max  0.782 mean  0.754
                          CPU CSR    min  3.310 max  3.324 mean  3.318
                          GPU 32 COO min  0.000 max  0.000 mean  0.000
                                 CSR min  0.000 max  0.000 mean  0.000
                          GPU 64 COO min  7.030 max  7.160 mean  7.100
                                 CSR min 15.730 max 18.530 mean 17.362
                          CPU PAR    min  1.340 max  1.451 mean  1.401
                          H          min 11.099 max 11.104 mean 11.102
bloweya.mtx
 Regular
                          CPU COO    min  0.727 max  1.815 mean  0.892
                          CPU CSR    min  2.867 max  2.936 mean  2.917
                          GPU 32 COO min  0.000 max  0.000 mean  0.000
                                 CSR min  0.000 max  0.000 mean  0.000
                          GPU 64 COO min  0.000 max  0.000 mean  0.000
                                 CSR min  0.000 max  0.000 mean  0.000
                          CPU PAR    min  1.680 max  1.751 mean  1.719
                          H          min  7.205 max  7.205 mean  7.205
 Row-Premute
                          CPU COO    min  0.678 max  1.483 mean  0.746
                          CPU CSR    min  2.311 max  2.326 mean  2.320
                          GPU 32 COO min  0.000 max  0.000 mean  0.000
                                 CSR min  0.000 max  0.000 mean  0.000
                          GPU 64 COO min  6.840 max  7.270 mean  6.930
                                 CSR min 15.650 max 16.800 mean 16.233
                          CPU PAR    min  1.649 max  1.730 mean  1.682
                          H          min 11.026 max 11.031 mean 11.029
 Row-Gradient
                          CPU COO    min  0.708 max  1.209 mean  0.779
                          CPU CSR    min  1.648 max  1.735 mean  1.709
                          GPU 32 COO min  0.000 max  0.000 mean  0.000
                                 CSR min  0.000 max  0.000 mean  0.000
                          GPU 64 COO min  6.920 max  7.080 mean  7.015
                                 CSR min 16.950 max 19.500 mean 17.794
                          CPU PAR    min  1.497 max  1.743 mean  1.608
                          H          min 10.298 max 10.304 mean 10.301
 Column-Gradient
                          CPU COO    min  0.709 max  1.536 mean  0.817
                          CPU CSR    min  1.705 max  1.753 mean  1.735
                          GPU 32 COO min  0.000 max  0.000 mean  0.000
                                 CSR min  0.000 max  0.000 mean  0.000
                          GPU 64 COO min  6.800 max  7.120 mean  6.865
                                 CSR min 15.480 max 17.710 mean 16.470
                          CPU PAR    min  1.446 max  1.718 mean  1.591
                          H          min 10.880 max 10.886 mean 10.883
 Row-Column-Permute
                          CPU COO    min  0.670 max  1.024 mean  0.706
                          CPU CSR    min  2.199 max  2.340 mean  2.326
                          GPU 32 COO min  0.000 max  0.000 mean  0.000
                                 CSR min  0.000 max  0.000 mean  0.000
                          GPU 64 COO min  6.880 max  6.980 mean  6.933
                                 CSR min 15.610 max 16.900 mean 16.227
                          CPU PAR    min  1.598 max  1.668 mean  1.632
                          H          min 11.025 max 11.032 mean 11.029
lp_osa_07.mtx
 Regular
                          CPU COO    min  0.715 max  1.798 mean  0.885
                          CPU CSR    min  2.495 max  2.551 mean  2.547
                          GPU 32 COO min  0.000 max  0.000 mean  0.000
                                 CSR min  0.000 max  0.000 mean  0.000
                          GPU 64 COO min  7.650 max  7.790 mean  7.718
                                 CSR min 16.390 max 18.350 mean 17.093
                          CPU PAR    min  0.963 max  1.012 mean  0.995
                          H          min  8.412 max  8.412 mean  8.412
 Row-Premute
                          CPU COO    min  0.720 max  2.078 mean  1.104
                          CPU CSR    min  2.656 max  2.679 mean  2.669
                          GPU 32 COO min  0.000 max  0.000 mean  0.000
                                 CSR min  0.000 max  0.000 mean  0.000
                          GPU 64 COO min  7.610 max  7.690 mean  7.647
                                 CSR min 15.910 max 17.210 mean 16.750
                          CPU PAR    min  0.890 max  0.940 mean  0.918
                          H          min  9.255 max  9.258 mean  9.256
 Row-Gradient
                          CPU COO    min  0.725 max  2.078 mean  1.041
                          CPU CSR    min  2.487 max  2.502 mean  2.495
                          GPU 32 COO min  0.000 max  0.000 mean  0.000
                                 CSR min  0.000 max  0.000 mean  0.000
                          GPU 64 COO min  7.570 max  7.730 mean  7.655
                                 CSR min 15.370 max 18.100 mean 16.803
                          CPU PAR    min  1.435 max  1.796 mean  1.592
                          H          min  8.637 max  8.678 mean  8.672
 Column-Gradient
                          CPU COO    min  0.724 max  1.990 mean  1.000
                          CPU CSR    min  2.425 max  2.477 mean  2.448
                          GPU 32 COO min  0.000 max  0.000 mean  0.000
                                 CSR min  0.000 max  0.000 mean  0.000
                          GPU 64 COO min  7.510 max  7.660 mean  7.596
                                 CSR min 14.410 max 16.290 mean 15.267
                          CPU PAR    min  1.238 max  1.774 mean  1.534
                          H          min  9.447 max  9.603 mean  9.576
 Row-Column-Permute
                          CPU COO    min  0.738 max  1.950 mean  1.071
                          CPU CSR    min  2.522 max  2.709 mean  2.675
                          GPU 32 COO min  0.000 max  0.000 mean  0.000
                                 CSR min  0.000 max  0.000 mean  0.000
                          GPU 64 COO min  7.600 max  7.690 mean  7.641
                                 CSR min 15.820 max 17.190 mean 16.572
                          CPU PAR    min  0.891 max  0.944 mean  0.924
                          H          min  9.255 max  9.258 mean  9.256
ex19.mtx
 Regular
                          CPU COO    min  0.732 max  1.837 mean  1.076
                          CPU CSR    min  2.563 max  2.586 mean  2.577
                          GPU 32 COO min  0.000 max  0.000 mean  0.000
                                 CSR min  0.000 max  0.000 mean  0.000
                          GPU 64 COO min 11.340 max 11.860 mean 11.441
                                 CSR min 36.010 max 40.960 mean 38.048
                          CPU PAR    min  2.019 max  2.204 mean  2.130
                          H          min  8.228 max  8.228 mean  8.228
 Row-Premute
                          CPU COO    min  0.718 max  0.751 mean  0.732
                          CPU CSR    min  2.488 max  2.507 mean  2.498
                          GPU 32 COO min  0.000 max  0.000 mean  0.000
                                 CSR min  0.000 max  0.000 mean  0.000
                          GPU 64 COO min 10.810 max 11.090 mean 10.949
                                 CSR min 24.860 max 26.410 mean 25.527
                          CPU PAR    min  1.978 max  2.290 mean  2.135
                          H          min 11.836 max 11.840 mean 11.838
 Row-Gradient
                          CPU COO    min  0.722 max  1.794 mean  0.769
                          CPU CSR    min  2.407 max  2.421 mean  2.416
                          GPU 32 COO min  0.000 max  0.000 mean  0.000
                                 CSR min  0.000 max  0.000 mean  0.000
                          GPU 64 COO min 11.210 max 11.480 mean 11.317
                                 CSR min 31.920 max 34.690 mean 33.246
                          CPU PAR    min  2.184 max  2.302 mean  2.232
                          H          min 10.742 max 10.757 mean 10.748
 Column-Gradient
                          CPU COO    min  0.720 max  0.916 mean  0.742
                          CPU CSR    min  2.395 max  2.410 mean  2.402
                          GPU 32 COO min  0.000 max  0.000 mean  0.000
                                 CSR min  0.000 max  0.000 mean  0.000
                          GPU 64 COO min 10.840 max 11.070 mean 10.946
                                 CSR min 24.340 max 26.140 mean 25.393
                          CPU PAR    min  2.184 max  2.272 mean  2.223
                          H          min 11.873 max 11.882 mean 11.878
 Row-Column-Permute
                          CPU COO    min  0.707 max  0.748 mean  0.714
                          CPU CSR    min  2.458 max  2.511 mean  2.506
                          GPU 32 COO min  0.000 max  0.000 mean  0.000
                                 CSR min  0.000 max  0.000 mean  0.000
                          GPU 64 COO min 10.880 max 11.070 mean 10.957
                                 CSR min 24.890 max 26.490 mean 25.642
                          CPU PAR    min  2.209 max  2.282 mean  2.240
                          H          min 11.834 max 11.840 mean 11.838
brainpc2.mtx
 Regular
                          CPU COO    min  0.732 max  0.751 mean  0.744
                          CPU CSR    min  2.885 max  2.916 mean  2.909
                          GPU 32 COO min  0.000 max  0.000 mean  0.000
                                 CSR min  0.000 max  0.000 mean  0.000
                          GPU 64 COO min  0.000 max  0.000 mean  0.000
                                 CSR min  0.000 max  0.000 mean  0.000
                          CPU PAR    min  1.276 max  1.299 mean  1.286
                          H          min  7.478 max  7.478 mean  7.478
 Row-Premute
                          CPU COO    min  0.727 max  0.855 mean  0.736
                          CPU CSR    min  2.385 max  2.411 mean  2.397
                          GPU 32 COO min  0.000 max  0.000 mean  0.000
                                 CSR min  0.000 max  0.000 mean  0.000
                          GPU 64 COO min  8.120 max  8.410 mean  8.206
                                 CSR min 18.670 max 19.960 mean 19.536
                          CPU PAR    min  1.293 max  1.340 mean  1.314
                          H          min  9.809 max  9.813 mean  9.811
 Row-Gradient
                          CPU COO    min  0.696 max  1.546 mean  0.785
                          CPU CSR    min  1.361 max  1.420 mean  1.411
                          GPU 32 COO min  0.000 max  0.000 mean  0.000
                                 CSR min  0.000 max  0.000 mean  0.000
                          GPU 64 COO min  8.190 max  8.550 mean  8.302
                                 CSR min 18.700 max 21.000 mean 19.890
                          CPU PAR    min  1.435 max  1.666 mean  1.549
                          H          min  9.721 max  9.727 mean  9.723
 Column-Gradient
                          CPU COO    min  0.698 max  1.467 mean  0.746
                          CPU CSR    min  1.377 max  1.423 mean  1.414
                          GPU 32 COO min  0.000 max  0.000 mean  0.000
                                 CSR min  0.000 max  0.000 mean  0.000
                          GPU 64 COO min  8.110 max  8.290 mean  8.187
                                 CSR min 18.090 max 20.190 mean 19.217
                          CPU PAR    min  1.345 max  1.681 mean  1.518
                          H          min 10.369 max 10.372 mean 10.370
 Row-Column-Permute
                          CPU COO    min  0.698 max  1.390 mean  0.788
                          CPU CSR    min  2.387 max  2.410 mean  2.399
                          GPU 32 COO min  0.000 max  0.000 mean  0.000
                                 CSR min  0.000 max  0.000 mean  0.000
                          GPU 64 COO min  8.120 max  8.260 mean  8.191
                                 CSR min 18.530 max 19.960 mean 19.307
                          CPU PAR    min  1.295 max  1.347 mean  1.319
                          H          min  9.809 max  9.813 mean  9.811
shermanACb.mtx
 Regular
                          CPU COO    min  0.712 max  1.201 mean  0.756
                          CPU CSR    min  1.558 max  1.601 mean  1.596
                          GPU 32 COO min  0.000 max  0.000 mean  0.000
                                 CSR min  0.000 max  0.000 mean  0.000
                          GPU 64 COO min  7.080 max  7.370 mean  7.184
                                 CSR min 17.580 max 19.480 mean 18.770
                          CPU PAR    min  1.286 max  1.511 mean  1.447
                          H          min  8.600 max  8.600 mean  8.600
 Row-Premute
                          CPU COO    min  0.689 max  0.890 mean  0.704
                          CPU CSR    min  1.600 max  1.630 mean  1.618
                          GPU 32 COO min  0.000 max  0.000 mean  0.000
                                 CSR min  0.000 max  0.000 mean  0.000
                          GPU 64 COO min  7.000 max  7.180 mean  7.061
                                 CSR min 15.760 max 17.240 mean 16.625
                          CPU PAR    min  1.296 max  1.419 mean  1.365
                          H          min 10.376 max 10.380 mean 10.379
 Row-Gradient
                          CPU COO    min  0.704 max  1.615 mean  0.806
                          CPU CSR    min  1.355 max  1.370 mean  1.362
                          GPU 32 COO min  0.000 max  0.000 mean  0.000
                                 CSR min  0.000 max  0.000 mean  0.000
                          GPU 64 COO min  7.020 max  7.160 mean  7.083
                                 CSR min  0.000 max 16.290 mean 15.076
                          CPU PAR    min  1.256 max  1.520 mean  1.405
                          H          min  9.915 max  9.925 mean  9.921
 Column-Gradient
                          CPU COO    min  0.702 max  1.626 mean  0.844
                          CPU CSR    min  1.327 max  1.374 mean  1.364
                          GPU 32 COO min  0.000 max  0.000 mean  0.000
                                 CSR min  0.000 max  0.000 mean  0.000
                          GPU 64 COO min  6.920 max  7.210 mean  7.030
                                 CSR min  0.000 max 15.260 mean 14.279
                          CPU PAR    min  1.283 max  1.531 mean  1.385
                          H          min 10.572 max 10.595 mean 10.590
 Row-Column-Permute
                          CPU COO    min  0.707 max  1.532 mean  0.924
                          CPU CSR    min  1.606 max  1.634 mean  1.624
                          GPU 32 COO min  0.000 max  0.000 mean  0.000
                                 CSR min  0.000 max  0.000 mean  0.000
                          GPU 64 COO min  6.970 max  7.110 mean  7.045
                                 CSR min 15.850 max 17.310 mean 16.783
                          CPU PAR    min  1.286 max  1.406 mean  1.357
                          H          min 10.377 max 10.382 mean 10.379
cvxqp3.mtx
 Regular
                          CPU COO    min  0.697 max  0.720 mean  0.712
                          CPU CSR    min  2.624 max  2.643 mean  2.638
                          GPU 32 COO min  0.000 max  0.000 mean  0.000
                                 CSR min  0.000 max  0.000 mean  0.000
                          GPU 64 COO min  6.060 max  6.220 mean  6.121
                                 CSR min 19.450 max 22.710 mean 21.277
                          CPU PAR    min  1.733 max  1.860 mean  1.804
                          H          min  8.646 max  8.646 mean  8.646
 Row-Premute
                          CPU COO    min  0.695 max  1.577 mean  0.894
                          CPU CSR    min  2.452 max  2.471 mean  2.464
                          GPU 32 COO min  0.000 max  0.000 mean  0.000
                                 CSR min  0.000 max  0.000 mean  0.000
                          GPU 64 COO min  5.870 max  6.060 mean  5.930
                                 CSR min 17.510 max 19.130 mean 18.516
                          CPU PAR    min  1.723 max  1.833 mean  1.774
                          H          min 11.028 max 11.033 mean 11.030
 Row-Gradient
                          CPU COO    min  0.693 max  1.523 mean  0.788
                          CPU CSR    min  1.287 max  1.305 mean  1.296
                          GPU 32 COO min  0.000 max  0.000 mean  0.000
                                 CSR min  0.000 max  0.000 mean  0.000
                          GPU 64 COO min  5.920 max  6.000 mean  5.962
                                 CSR min 16.810 max 18.410 mean 17.561
                          CPU PAR    min  1.378 max  1.485 mean  1.429
                          H          min 11.061 max 11.069 mean 11.064
 Column-Gradient
                          CPU COO    min  0.693 max  1.521 mean  0.772
                          CPU CSR    min  1.291 max  1.302 mean  1.297
                          GPU 32 COO min  0.000 max  0.000 mean  0.000
                                 CSR min  0.000 max  0.000 mean  0.000
                          GPU 64 COO min  5.900 max  6.060 mean  5.960
                                 CSR min 16.620 max 18.330 mean 17.592
                          CPU PAR    min  1.372 max  1.464 mean  1.409
                          H          min 11.127 max 11.135 mean 11.130
 Row-Column-Permute
                          CPU COO    min  0.704 max  1.503 mean  0.875
                          CPU CSR    min  2.447 max  2.468 mean  2.459
                          GPU 32 COO min  0.000 max  0.000 mean  0.000
                                 CSR min  0.000 max  0.000 mean  0.000
                          GPU 64 COO min  5.880 max  5.980 mean  5.931
                                 CSR min 17.550 max 19.140 mean 18.227
                          CPU PAR    min  1.639 max  1.743 mean  1.704
                          H          min 11.028 max 11.035 mean 11.030
case9.mtx
 Regular
                          CPU COO    min  0.721 max  1.800 mean  1.177
                          CPU CSR    min  3.021 max  3.046 mean  3.036
                          GPU 32 COO min  0.000 max  0.000 mean  0.000
                                 CSR min  0.000 max  0.000 mean  0.000
                          GPU 64 COO min  0.000 max  0.000 mean  0.000
                                 CSR min  0.000 max  0.000 mean  0.000
                          CPU PAR    min  1.508 max  1.605 mean  1.573
                          H          min  7.380 max  7.380 mean  7.380
 Row-Premute
                          CPU COO    min  0.724 max  1.100 mean  0.765
                          CPU CSR    min  2.581 max  2.626 mean  2.609
                          GPU 32 COO min  0.000 max  0.000 mean  0.000
                                 CSR min  0.000 max  0.000 mean  0.000
                          GPU 64 COO min  7.170 max  7.340 mean  7.253
                                 CSR min 17.360 max 18.500 mean 18.014
                          CPU PAR    min  1.494 max  1.607 mean  1.558
                          H          min 10.043 max 10.047 mean 10.044
 Row-Gradient
                          CPU COO    min  0.716 max  1.701 mean  0.804
                          CPU CSR    min  1.824 max  1.840 mean  1.832
                          GPU 32 COO min  0.000 max  0.000 mean  0.000
                                 CSR min  0.000 max  0.000 mean  0.000
                          GPU 64 COO min  7.220 max  7.510 mean  7.303
                                 CSR min 17.540 max 20.710 mean 19.302
                          CPU PAR    min  1.384 max  1.593 mean  1.526
                          H          min  9.681 max  9.706 mean  9.694
 Column-Gradient
                          CPU COO    min  0.711 max  1.029 mean  0.746
                          CPU CSR    min  1.817 max  1.834 mean  1.827
                          GPU 32 COO min  0.000 max  0.000 mean  0.000
                                 CSR min  0.000 max  0.000 mean  0.000
                          GPU 64 COO min  7.110 max  7.270 mean  7.193
                                 CSR min 16.530 max 18.590 mean 17.574
                          CPU PAR    min  1.390 max  1.574 mean  1.511
                          H          min 10.612 max 10.659 mean 10.634
 Row-Column-Permute
                          CPU COO    min  0.719 max  1.391 mean  0.756
                          CPU CSR    min  2.546 max  2.625 mean  2.611
                          GPU 32 COO min  0.000 max  0.000 mean  0.000
                                 CSR min  0.000 max  0.000 mean  0.000
                          GPU 64 COO min  7.190 max  7.320 mean  7.248
                                 CSR min 17.500 max 18.640 mean 18.040
                          CPU PAR    min  1.465 max  1.573 mean  1.533
                          H          min 10.041 max 10.046 mean 10.044
TSOPF_FS_b9_c6.mtx
 Regular
                          CPU COO    min  0.705 max  0.734 mean  0.718
                          CPU CSR    min  3.028 max  3.052 mean  3.045
                          GPU 32 COO min  0.000 max  0.000 mean  0.000
                                 CSR min  0.000 max  0.000 mean  0.000
                          GPU 64 COO min  0.000 max  0.000 mean  0.000
                                 CSR min  0.000 max  0.000 mean  0.000
                          CPU PAR    min  1.528 max  1.602 mean  1.568
                          H          min  7.380 max  7.380 mean  7.380
 Row-Premute
                          CPU COO    min  0.733 max  1.640 mean  0.777
                          CPU CSR    min  2.450 max  2.543 mean  2.525
                          GPU 32 COO min  0.000 max  0.000 mean  0.000
                                 CSR min  0.000 max  0.000 mean  0.000
                          GPU 64 COO min  7.200 max  7.320 mean  7.268
                                 CSR min 17.420 max 18.540 mean 18.102
                          CPU PAR    min  1.474 max  1.595 mean  1.546
                          H          min 10.042 max 10.046 mean 10.044
 Row-Gradient
                          CPU COO    min  0.712 max  0.926 mean  0.750
                          CPU CSR    min  1.819 max  1.846 mean  1.832
                          GPU 32 COO min  0.000 max  0.000 mean  0.000
                                 CSR min  0.000 max  0.000 mean  0.000
                          GPU 64 COO min  7.210 max  7.370 mean  7.298
                                 CSR min 17.550 max 20.740 mean 19.089
                          CPU PAR    min  1.256 max  1.554 mean  1.495
                          H          min  9.666 max  9.704 mean  9.690
 Column-Gradient
                          CPU COO    min  0.710 max  1.690 mean  0.791
                          CPU CSR    min  1.813 max  1.836 mean  1.830
                          GPU 32 COO min  0.000 max  0.000 mean  0.000
                                 CSR min  0.000 max  0.000 mean  0.000
                          GPU 64 COO min  7.130 max  7.310 mean  7.211
                                 CSR min 16.550 max 18.690 mean 17.617
                          CPU PAR    min  1.385 max  1.539 mean  1.506
                          H          min 10.611 max 10.659 mean 10.634
 Row-Column-Permute
                          CPU COO    min  0.709 max  1.531 mean  0.963
                          CPU CSR    min  2.506 max  2.648 mean  2.622
                          GPU 32 COO min  0.000 max  0.000 mean  0.000
                                 CSR min  0.000 max  0.000 mean  0.000
                          GPU 64 COO min  7.140 max  7.330 mean  7.244
                                 CSR min 17.410 max 18.520 mean 18.148
                          CPU PAR    min  1.466 max  1.574 mean  1.528
                          H          min 10.041 max 10.046 mean 10.044
OPF_6000.mtx
 Regular
                          CPU COO    min  0.714 max  0.731 mean  0.720
                          CPU CSR    min  2.667 max  2.770 mean  2.720
                          GPU 32 COO min  0.000 max  0.000 mean  0.000
                                 CSR min  0.000 max  0.000 mean  0.000
                          GPU 64 COO min 12.310 max 12.550 mean 12.425
                                 CSR min 39.860 max 43.770 mean 42.075
                          CPU PAR    min  1.735 max  1.945 mean  1.845
                          H          min  8.799 max  8.799 mean  8.799
 Row-Premute
                          CPU COO    min  0.689 max  0.710 mean  0.695
                          CPU CSR    min  2.358 max  2.413 mean  2.392
                          GPU 32 COO min  0.000 max  0.000 mean  0.000
                                 CSR min  0.000 max  0.000 mean  0.000
                          GPU 64 COO min 11.430 max 11.770 mean 11.549
                                 CSR min 24.470 max 25.580 mean 24.785
                          CPU PAR    min  1.758 max  1.896 mean  1.829
                          H          min 11.872 max 11.877 mean 11.875
 Row-Gradient
                          CPU COO    min  0.716 max  0.775 mean  0.739
                          CPU CSR    min  1.651 max  1.689 mean  1.675
                          GPU 32 COO min  0.000 max  0.000 mean  0.000
                                 CSR min  0.000 max  0.000 mean  0.000
                          GPU 64 COO min 12.100 max 12.410 mean 12.205
                                 CSR min 31.670 max 34.910 mean 33.370
                          CPU PAR    min  2.079 max  2.286 mean  2.207
                          H          min 11.111 max 11.116 mean 11.113
 Column-Gradient
                          CPU COO    min  0.715 max  1.021 mean  0.743
                          CPU CSR    min  1.655 max  1.674 mean  1.666
                          GPU 32 COO min  0.000 max  0.000 mean  0.000
                                 CSR min  0.000 max  0.000 mean  0.000
                          GPU 64 COO min 11.340 max 11.560 mean 11.463
                                 CSR min 23.770 max 25.470 mean 24.489
                          CPU PAR    min  2.056 max  2.172 mean  2.118
                          H          min 12.040 max 12.047 mean 12.043
 Row-Column-Permute
                          CPU COO    min  0.677 max  0.785 mean  0.687
                          CPU CSR    min  2.325 max  2.434 mean  2.369
                          GPU 32 COO min  0.000 max  0.000 mean  0.000
                                 CSR min  0.000 max  0.000 mean  0.000
                          GPU 64 COO min 11.450 max 11.650 mean 11.538
                                 CSR min 24.330 max 25.560 mean 25.008
                          CPU PAR    min  1.631 max  1.776 mean  1.709
                          H          min 11.873 max 11.877 mean 11.875
OPF_3754.mtx
 Regular
                          CPU COO    min  0.726 max  0.774 mean  0.747
                          CPU CSR    min  2.898 max  2.919 mean  2.908
                          GPU 32 COO min  0.000 max  0.000 mean  0.000
                                 CSR min  0.000 max  0.000 mean  0.000
                          GPU 64 COO min  7.680 max  7.820 mean  7.766
                                 CSR min 25.070 max 29.030 mean 26.756
                          CPU PAR    min  1.437 max  1.508 mean  1.471
                          H          min  8.393 max  8.393 mean  8.393
 Row-Premute
                          CPU COO    min  0.714 max  1.574 mean  0.817
                          CPU CSR    min  2.686 max  2.711 mean  2.699
                          GPU 32 COO min  0.000 max  0.000 mean  0.000
                                 CSR min  0.000 max  0.000 mean  0.000
                          GPU 64 COO min  7.410 max  7.570 mean  7.484
                                 CSR min 19.600 max 21.190 mean 20.307
                          CPU PAR    min  1.443 max  1.505 mean  1.469
                          H          min 11.267 max 11.272 mean 11.269
 Row-Gradient
                          CPU COO    min  0.723 max  1.232 mean  0.775
                          CPU CSR    min  1.672 max  1.691 mean  1.685
                          GPU 32 COO min  0.000 max  0.000 mean  0.000
                                 CSR min  0.000 max  0.000 mean  0.000
                          GPU 64 COO min  7.600 max  7.760 mean  7.716
                                 CSR min 23.160 max 25.590 mean 24.304
                          CPU PAR    min  1.675 max  1.736 mean  1.703
                          H          min 10.463 max 10.472 mean 10.468
 Column-Gradient
                          CPU COO    min  0.726 max  1.431 mean  0.778
                          CPU CSR    min  1.671 max  1.685 mean  1.679
                          GPU 32 COO min  0.000 max  0.000 mean  0.000
                                 CSR min  0.000 max  0.000 mean  0.000
                          GPU 64 COO min  7.410 max  7.530 mean  7.467
                                 CSR min 18.140 max 20.350 mean 19.315
                          CPU PAR    min  1.650 max  1.736 mean  1.699
                          H          min 11.393 max 11.401 mean 11.397
 Row-Column-Permute
                          CPU COO    min  0.711 max  1.458 mean  0.751
                          CPU CSR    min  2.678 max  2.717 mean  2.700
                          GPU 32 COO min  0.000 max  0.000 mean  0.000
                                 CSR min  0.000 max  0.000 mean  0.000
                          GPU 64 COO min  7.400 max  7.540 mean  7.471
                                 CSR min 19.560 max 21.150 mean 20.453
                          CPU PAR    min  1.440 max  1.499 mean  1.467
                          H          min 11.266 max 11.272 mean 11.269
c-47.mtx
 Regular
                          CPU COO    min  0.754 max  1.829 mean  1.204
                          CPU CSR    min  2.610 max  2.624 mean  2.618
                          GPU 32 COO min  0.000 max  0.000 mean  0.000
                                 CSR min  0.000 max  0.000 mean  0.000
                          GPU 64 COO min  9.530 max  9.870 mean  9.640
                                 CSR min 23.990 max 25.910 mean 24.992
                          CPU PAR    min  1.311 max  1.380 mean  1.357
                          H          min  8.364 max  8.364 mean  8.364
 Row-Premute
                          CPU COO    min  0.740 max  0.885 mean  0.755
                          CPU CSR    min  2.574 max  2.611 mean  2.597
                          GPU 32 COO min  0.000 max  0.000 mean  0.000
                                 CSR min  0.000 max  0.000 mean  0.000
                          GPU 64 COO min  9.320 max  9.510 mean  9.397
                                 CSR min 19.960 max 21.190 mean 20.696
                          CPU PAR    min  1.303 max  1.371 mean  1.345
                          H          min 10.059 max 10.062 mean 10.061
 Row-Gradient
                          CPU COO    min  0.723 max  0.984 mean  0.753
                          CPU CSR    min  1.781 max  1.809 mean  1.803
                          GPU 32 COO min  0.000 max  0.000 mean  0.000
                                 CSR min  0.000 max  0.000 mean  0.000
                          GPU 64 COO min  9.380 max  9.660 mean  9.464
                                 CSR min 15.770 max 19.090 mean 18.037
                          CPU PAR    min  1.775 max  1.924 mean  1.868
                          H          min 10.205 max 10.233 mean 10.219
 Column-Gradient
                          CPU COO    min  0.715 max  0.926 mean  0.757
                          CPU CSR    min  1.729 max  1.802 mean  1.791
                          GPU 32 COO min  0.000 max  0.000 mean  0.000
                                 CSR min  0.000 max  0.000 mean  0.000
                          GPU 64 COO min  9.080 max  9.270 mean  9.158
                                 CSR min 13.980 max 15.780 mean 14.938
                          CPU PAR    min  1.751 max  1.906 mean  1.846
                          H          min 11.213 max 11.232 mean 11.222
 Row-Column-Permute
                          CPU COO    min  0.732 max  1.598 mean  0.785
                          CPU CSR    min  2.594 max  2.602 mean  2.599
                          GPU 32 COO min  0.000 max  0.000 mean  0.000
                                 CSR min  0.000 max  0.000 mean  0.000
                          GPU 64 COO min  9.340 max  9.460 mean  9.394
                                 CSR min 19.950 max 21.500 mean 20.544
                          CPU PAR    min  1.326 max  1.374 mean  1.354
                          H          min 10.059 max 10.062 mean 10.061
mhd4800a.mtx
 Regular
                          CPU COO    min  0.759 max  0.795 mean  0.780
                          CPU CSR    min  2.479 max  2.565 mean  2.557
                          GPU 32 COO min  0.000 max  0.000 mean  0.000
                                 CSR min  0.000 max  0.000 mean  0.000
                          GPU 64 COO min  5.490 max  5.650 mean  5.552
                                 CSR min 16.700 max 19.460 mean 18.004
                          CPU PAR    min  1.456 max  1.523 mean  1.492
                          H          min  7.132 max  7.132 mean  7.132
 Row-Premute
                          CPU COO    min  0.695 max  0.943 mean  0.726
                          CPU CSR    min  2.480 max  2.488 mean  2.485
                          GPU 32 COO min  0.000 max  0.000 mean  0.000
                                 CSR min  0.000 max  0.000 mean  0.000
                          GPU 64 COO min  5.410 max  5.490 mean  5.453
                                 CSR min 15.700 max 17.520 mean 16.678
                          CPU PAR    min  1.422 max  1.514 mean  1.474
                          H          min 10.959 max 10.966 mean 10.963
 Row-Gradient
                          CPU COO    min  0.723 max  2.029 mean  0.990
                          CPU CSR    min  2.411 max  2.427 mean  2.421
                          GPU 32 COO min  0.000 max  0.000 mean  0.000
                                 CSR min  0.000 max  0.000 mean  0.000
                          GPU 64 COO min  5.490 max  5.560 mean  5.534
                                 CSR min 16.350 max 19.560 mean 17.784
                          CPU PAR    min  1.441 max  1.509 mean  1.477
                          H          min  9.512 max  9.526 mean  9.520
 Column-Gradient
                          CPU COO    min  0.721 max  1.802 mean  0.871
                          CPU CSR    min  2.393 max  2.408 mean  2.404
                          GPU 32 COO min  0.000 max  0.000 mean  0.000
                                 CSR min  0.000 max  0.000 mean  0.000
                          GPU 64 COO min  5.410 max  5.480 mean  5.453
                                 CSR min 15.680 max 17.870 mean 16.540
                          CPU PAR    min  1.429 max  1.488 mean  1.468
                          H          min 10.931 max 10.945 mean 10.938
 Row-Column-Permute
                          CPU COO    min  0.728 max  1.646 mean  1.037
                          CPU CSR    min  2.472 max  2.488 mean  2.480
                          GPU 32 COO min  0.000 max  0.000 mean  0.000
                                 CSR min  0.000 max  0.000 mean  0.000
                          GPU 64 COO min  5.410 max  5.480 mean  5.449
                                 CSR min 15.760 max 17.560 mean 16.654
                          CPU PAR    min  1.428 max  1.513 mean  1.474
                          H          min 10.959 max 10.967 mean 10.963
gen4.mtx
 Regular
                          CPU COO    min  0.737 max  1.977 mean  1.431
                          CPU CSR    min  2.674 max  2.688 mean  2.681
                          GPU 32 COO min  0.000 max  0.000 mean  0.000
                                 CSR min  0.000 max  0.000 mean  0.000
                          GPU 64 COO min  5.900 max  6.000 mean  5.954
                                 CSR min 13.650 max 15.410 mean 14.657
                          CPU PAR    min  1.468 max  1.521 mean  1.491
                          H          min  9.234 max  9.234 mean  9.234
 Row-Premute
                          CPU COO    min  0.740 max  2.048 mean  1.121
                          CPU CSR    min  2.777 max  2.798 mean  2.790
                          GPU 32 COO min  0.000 max  0.000 mean  0.000
                                 CSR min  0.000 max  0.000 mean  0.000
                          GPU 64 COO min  5.910 max  5.970 mean  5.944
                                 CSR min 13.700 max 15.370 mean 14.541
                          CPU PAR    min  1.468 max  1.546 mean  1.502
                          H          min 10.250 max 10.255 mean 10.252
 Row-Gradient
                          CPU COO    min  0.740 max  1.790 mean  0.994
                          CPU CSR    min  2.663 max  2.682 mean  2.674
                          GPU 32 COO min  0.000 max  0.000 mean  0.000
                                 CSR min  0.000 max  0.000 mean  0.000
                          GPU 64 COO min  5.890 max  6.160 mean  5.946
                                 CSR min 13.780 max 17.520 mean 15.601
                          CPU PAR    min  1.479 max  1.619 mean  1.569
                          H          min  9.939 max  9.955 mean  9.948
 Column-Gradient
                          CPU COO    min  0.743 max  1.991 mean  0.981
                          CPU CSR    min  2.620 max  2.654 mean  2.646
                          GPU 32 COO min  0.000 max  0.000 mean  0.000
                                 CSR min  0.000 max  0.000 mean  0.000
                          GPU 64 COO min  5.840 max  5.910 mean  5.885
                                 CSR min 13.130 max 17.040 mean 15.008
                          CPU PAR    min  1.477 max  1.607 mean  1.559
                          H          min 10.858 max 10.876 mean 10.864
 Row-Column-Permute
                          CPU COO    min  0.742 max  2.010 mean  1.124
                          CPU CSR    min  2.789 max  2.800 mean  2.795
                          GPU 32 COO min  0.000 max  0.000 mean  0.000
                                 CSR min  0.000 max  0.000 mean  0.000
                          GPU 64 COO min  5.900 max  5.980 mean  5.941
                                 CSR min 13.640 max 15.410 mean 14.556
                          CPU PAR    min  1.462 max  1.540 mean  1.504
                          H          min 10.250 max 10.253 mean 10.252
Maragal_6.mtx
 Regular
                          CPU COO    min  0.725 max  0.741 mean  0.729
                          CPU CSR    min  2.345 max  2.409 mean  2.372
                          GPU 32 COO min  0.000 max  0.000 mean  0.000
                                 CSR min  0.000 max  0.000 mean  0.000
                          GPU 64 COO min 18.200 max 18.770 mean 18.357
                                 CSR min 38.310 max 40.240 mean 39.477
                          CPU PAR    min  0.789 max  0.813 mean  0.797
                          H          min  9.930 max  9.930 mean  9.930
 Row-Premute
                          CPU COO    min  0.709 max  0.779 mean  0.715
                          CPU CSR    min  2.675 max  2.715 mean  2.696
                          GPU 32 COO min  0.000 max  0.000 mean  0.000
                                 CSR min  0.000 max  0.000 mean  0.000
                          GPU 64 COO min 17.810 max 18.030 mean 17.935
                                 CSR min 29.650 max 30.580 mean 30.109
                          CPU PAR    min  0.857 max  0.940 mean  0.904
                          H          min 10.777 max 10.779 mean 10.778
 Row-Gradient
                          CPU COO    min  0.710 max  1.566 mean  0.755
                          CPU CSR    min  2.042 max  2.159 mean  2.120
                          GPU 32 COO min  0.000 max  0.000 mean  0.000
                                 CSR min  0.000 max  0.000 mean  0.000
                          GPU 64 COO min 18.460 max 18.960 mean 18.665
                                 CSR min 25.650 max 27.330 mean 26.549
                          CPU PAR    min  2.257 max  2.612 mean  2.416
                          H          min 11.251 max 11.301 mean 11.285
 Column-Gradient
                          CPU COO    min  0.711 max  0.743 mean  0.725
                          CPU CSR    min  2.036 max  2.161 mean  2.110
                          GPU 32 COO min  0.000 max  0.000 mean  0.000
                                 CSR min  0.000 max  0.000 mean  0.000
                          GPU 64 COO min 17.840 max 18.860 mean 18.149
                                 CSR min 19.410 max 20.690 mean 20.066
                          CPU PAR    min  2.174 max  2.546 mean  2.349
                          H          min 12.011 max 12.072 mean 12.052
 Row-Column-Permute
                          CPU COO    min  0.712 max  0.971 mean  0.737
                          CPU CSR    min  2.732 max  2.751 mean  2.743
                          GPU 32 COO min  0.000 max  0.000 mean  0.000
                                 CSR min  0.000 max  0.000 mean  0.000
                          GPU 64 COO min 17.720 max 18.070 mean 17.911
                                 CSR min 29.600 max 30.500 mean 29.961
                          CPU PAR    min  0.827 max  0.954 mean  0.913
                          H          min 10.776 max 10.778 mean 10.777
aft01.mtx
 Regular
                          CPU COO    min  0.735 max  2.079 mean  1.069
                          CPU CSR    min  3.132 max  3.154 mean  3.145
                          GPU 32 COO min  0.000 max  0.000 mean  0.000
                                 CSR min  0.000 max  0.000 mean  0.000
                          GPU 64 COO min  6.390 max  6.610 mean  6.457
                                 CSR min 19.990 max 23.250 mean 21.820
                          CPU PAR    min  1.746 max  1.865 mean  1.812
                          H          min  7.811 max  7.811 mean  7.811
 Row-Premute
                          CPU COO    min  0.714 max  1.648 mean  0.840
                          CPU CSR    min  2.864 max  2.892 mean  2.883
                          GPU 32 COO min  0.000 max  0.000 mean  0.000
                                 CSR min  0.000 max  0.000 mean  0.000
                          GPU 64 COO min  6.280 max  6.380 mean  6.329
                                 CSR min 17.980 max 19.700 mean 19.105
                          CPU PAR    min  1.729 max  1.850 mean  1.782
                          H          min 11.162 max 11.168 mean 11.165
 Row-Gradient
                          CPU COO    min  0.735 max  1.806 mean  0.878
                          CPU CSR    min  2.706 max  2.744 mean  2.726
                          GPU 32 COO min  0.000 max  0.000 mean  0.000
                                 CSR min  0.000 max  0.000 mean  0.000
                          GPU 64 COO min  6.390 max  6.500 mean  6.433
                                 CSR min 19.780 max 22.870 mean 20.936
                          CPU PAR    min  1.710 max  1.865 mean  1.785
                          H          min 10.251 max 10.267 mean 10.257
 Column-Gradient
                          CPU COO    min  0.728 max  1.792 mean  0.986
                          CPU CSR    min  2.521 max  2.720 mean  2.703
                          GPU 32 COO min  0.000 max  0.000 mean  0.000
                                 CSR min  0.000 max  0.000 mean  0.000
                          GPU 64 COO min  6.280 max  6.370 mean  6.327
                                 CSR min 18.000 max 19.720 mean 19.040
                          CPU PAR    min  1.649 max  1.741 mean  1.702
                          H          min 11.113 max 11.121 mean 11.117
 Row-Column-Permute
                          CPU COO    min  0.714 max  1.525 mean  0.957
                          CPU CSR    min  2.876 max  2.892 mean  2.884
                          GPU 32 COO min  0.000 max  0.000 mean  0.000
                                 CSR min  0.000 max  0.000 mean  0.000
                          GPU 64 COO min  6.280 max  6.370 mean  6.322
                                 CSR min 17.960 max 19.670 mean 18.670
                          CPU PAR    min  1.667 max  1.754 mean  1.710
                          H          min 11.162 max 11.168 mean 11.165
TSOPF_RS_b39_c7.mtx
 Regular
                          CPU COO    min  0.771 max  0.793 mean  0.780
                          CPU CSR    min  3.219 max  3.232 mean  3.227
                          GPU 32 COO min  0.000 max  0.000 mean  0.000
                                 CSR min  0.000 max  0.000 mean  0.000
                          GPU 64 COO min 11.070 max 11.200 mean 11.142
                                 CSR min 37.050 max 42.100 mean 39.040
                          CPU PAR    min  1.910 max  2.027 mean  1.982
                          H          min  7.304 max  7.304 mean  7.304
 Row-Premute
                          CPU COO    min  0.701 max  0.722 mean  0.707
                          CPU CSR    min  2.931 max  2.952 mean  2.942
                          GPU 32 COO min  0.000 max  0.000 mean  0.000
                                 CSR min  0.000 max  0.000 mean  0.000
                          GPU 64 COO min 10.860 max 11.030 mean 10.928
                                 CSR min 28.730 max 30.880 mean 29.483
                          CPU PAR    min  1.760 max  1.922 mean  1.851
                          H          min 10.537 max 10.541 mean 10.539
 Row-Gradient
                          CPU COO    min  0.747 max  0.808 mean  0.757
                          CPU CSR    min  2.606 max  2.648 mean  2.624
                          GPU 32 COO min  0.000 max  0.000 mean  0.000
                                 CSR min  0.000 max  0.000 mean  0.000
                          GPU 64 COO min 10.850 max 11.120 mean 10.999
                                 CSR min 33.910 max 37.600 mean 35.909
                          CPU PAR    min  2.154 max  2.245 mean  2.203
                          H          min  9.636 max  9.646 mean  9.642
 Column-Gradient
                          CPU COO    min  0.718 max  1.693 mean  0.802
                          CPU CSR    min  2.502 max  2.585 mean  2.547
                          GPU 32 COO min  0.000 max  0.000 mean  0.000
                                 CSR min  0.000 max  0.000 mean  0.000
                          GPU 64 COO min 10.700 max 10.990 mean 10.804
                                 CSR min 27.230 max 29.380 mean 28.488
                          CPU PAR    min  2.128 max  2.227 mean  2.172
                          H          min 11.131 max 11.222 mean 11.208
 Row-Column-Permute
                          CPU COO    min  0.709 max  0.726 mean  0.716
                          CPU CSR    min  2.917 max  2.958 mean  2.940
                          GPU 32 COO min  0.000 max  0.000 mean  0.000
                                 CSR min  0.000 max  0.000 mean  0.000
                          GPU 64 COO min 10.840 max 11.030 mean 10.930
                                 CSR min 28.780 max 30.810 mean 29.578
                          CPU PAR    min  1.757 max  1.834 mean  1.792
                          H          min 10.537 max 10.540 mean 10.539
\end{verbatim}
 \section{Ellesmere}
\label{sec:ellesmere}
{\tiny

\begin{verbatim}
aft01.mtx
 Regular
                          GPU 64 COO min  4.080 max  4.280 mean  4.186
                                 CSR min  9.660 max 12.660 mean 11.485
                          H          min  7.811 max  7.811 mean  7.811
 Row-Premute
                          GPU 64 COO min  3.860 max  4.090 mean  4.001
                                 CSR min  9.520 max 10.340 mean  9.936
                          H          min 11.161 max 11.167 mean 11.165
 Row-Gradient
                          GPU 64 COO min  4.010 max  4.240 mean  4.135
                                 CSR min  5.890 max 11.350 mean  6.882
                          H          min 10.246 max 10.262 mean 10.256
 Column-Gradient
                          GPU 64 COO min  3.850 max  4.100 mean  4.012
                                 CSR min  5.460 max  8.790 mean  6.005
                          H          min 11.112 max 11.122 mean 11.117
 Row-Column-Permute
                          GPU 64 COO min  3.850 max  4.080 mean  3.990
                                 CSR min  5.420 max  6.760 mean  5.977
                          H          min 11.162 max 11.169 mean 11.165
bloweya.mtx
 Regular
                          GPU 64 COO min  0.000 max  0.000 mean  0.000
                                 CSR min  0.000 max  0.000 mean  0.000
                          H          min  7.205 max  7.205 mean  7.205
 Row-Premute
                          GPU 64 COO min  3.800 max  3.940 mean  3.875
                                 CSR min  3.710 max  4.570 mean  4.399
                          H          min 11.025 max 11.031 mean 11.028
 Row-Gradient
                          GPU 64 COO min  3.800 max  4.120 mean  3.962
                                 CSR min  4.340 max  4.670 mean  4.546
                          H          min 10.296 max 10.307 mean 10.300
 Column-Gradient
                          GPU 64 COO min  3.880 max  4.100 mean  3.978
                                 CSR min  4.240 max  4.570 mean  4.412
                          H          min 10.881 max 10.886 mean 10.883
 Row-Column-Permute
                          GPU 64 COO min  3.800 max  3.980 mean  3.885
                                 CSR min  4.130 max  4.540 mean  4.399
                          H          min 11.025 max 11.033 mean 11.029
brainpc2.mtx
 Regular
                          GPU 64 COO min  0.000 max  0.000 mean  0.000
                                 CSR min  0.000 max  0.000 mean  0.000
                          H          min  7.478 max  7.478 mean  7.478
 Row-Premute
                          GPU 64 COO min  3.840 max  6.750 mean  4.110
                                 CSR min  4.260 max  4.500 mean  4.437
                          H          min  9.809 max  9.813 mean  9.811
 Row-Gradient
                          GPU 64 COO min  0.640 max  4.030 mean  3.864
                                 CSR min  4.270 max  4.470 mean  4.383
                          H          min  9.722 max  9.727 mean  9.724
 Column-Gradient
                          GPU 64 COO min  0.640 max  4.070 mean  3.898
                                 CSR min  4.230 max  4.500 mean  4.386
                          H          min 10.368 max 10.372 mean 10.370
 Row-Column-Permute
                          GPU 64 COO min  3.980 max  4.110 mean  4.027
                                 CSR min  4.320 max  4.490 mean  4.437
                          H          min  9.809 max  9.813 mean  9.811
c-47.mtx
 Regular
                          GPU 64 COO min  3.980 max  4.080 mean  4.026
                                 CSR min  4.760 max  4.850 mean  4.812
                          H          min  8.364 max  8.364 mean  8.364
 Row-Premute
                          GPU 64 COO min  3.880 max  4.010 mean  3.942
                                 CSR min  4.040 max  4.900 mean  4.807
                          H          min 10.059 max 10.063 mean 10.061
 Row-Gradient
                          GPU 64 COO min  3.900 max  4.050 mean  3.976
                                 CSR min  4.380 max  4.740 mean  4.630
                          H          min 10.201 max 10.228 mean 10.214
 Column-Gradient
                          GPU 64 COO min  3.860 max  3.990 mean  3.936
                                 CSR min  4.350 max  4.610 mean  4.525
                          H          min 11.204 max 11.241 mean 11.222
 Row-Column-Permute
                          GPU 64 COO min  3.890 max  4.020 mean  3.953
                                 CSR min  4.490 max  4.920 mean  4.840
                          H          min 10.058 max 10.063 mean 10.061
case9.mtx
 Regular
                          GPU 64 COO min  0.000 max  0.000 mean  0.000
                                 CSR min  0.000 max  0.000 mean  0.000
                          H          min  7.380 max  7.380 mean  7.380
 Row-Premute
                          GPU 64 COO min  4.820 max  4.940 mean  4.859
                                 CSR min  5.080 max  6.520 mean  6.342
                          H          min 10.042 max 10.047 mean 10.044
 Row-Gradient
                          GPU 64 COO min  4.810 max  4.940 mean  4.876
                                 CSR min  6.100 max  6.560 mean  6.307
                          H          min  9.681 max  9.704 mean  9.694
 Column-Gradient
                          GPU 64 COO min  4.810 max  4.930 mean  4.869
                                 CSR min  4.820 max  6.460 mean  6.208
                          H          min 10.554 max 10.661 mean 10.638
 Row-Column-Permute
                          GPU 64 COO min  4.810 max  4.940 mean  4.864
                                 CSR min  5.930 max  6.520 mean  6.379
                          H          min 10.041 max 10.047 mean 10.044
cvxqp3.mtx
 Regular
                          GPU 64 COO min  3.350 max  3.590 mean  3.483
                                 CSR min  5.430 max  9.260 mean  8.333
                          H          min  8.646 max  8.646 mean  8.646
 Row-Premute
                          GPU 64 COO min  3.230 max  3.480 mean  3.371
                                 CSR min  7.560 max  8.220 mean  7.900
                          H          min 11.027 max 11.033 mean 11.030
 Row-Gradient
                          GPU 64 COO min  3.240 max  3.510 mean  3.396
                                 CSR min  6.990 max  7.890 mean  7.574
                          H          min 11.060 max 11.069 mean 11.064
 Column-Gradient
                          GPU 64 COO min  3.240 max  3.480 mean  3.374
                                 CSR min  6.980 max  7.900 mean  7.557
                          H          min 11.126 max 11.134 mean 11.130
 Row-Column-Permute
                          GPU 64 COO min  3.110 max  3.470 mean  3.365
                                 CSR min  4.810 max  8.210 mean  7.742
                          H          min 11.026 max 11.032 mean 11.030
ex19.mtx
 Regular
                          GPU 64 COO min  2.450 max  2.610 mean  2.564
                                 CSR min  4.490 max  4.760 mean  4.714
                          H          min  8.228 max  8.228 mean  8.228
 Row-Premute
                          GPU 64 COO min  2.000 max  2.040 mean  2.021
                                 CSR min  4.640 max  4.780 mean  4.733
                          H          min 11.835 max 11.840 mean 11.838
 Row-Gradient
                          GPU 64 COO min  2.240 max  2.390 mean  2.329
                                 CSR min  4.570 max  4.850 mean  4.807
                          H          min 10.742 max 10.752 mean 10.747
 Column-Gradient
                          GPU 64 COO min  2.010 max  2.050 mean  2.034
                                 CSR min  4.570 max  4.760 mean  4.701
                          H          min 11.872 max 11.881 mean 11.878
 Row-Column-Permute
                          GPU 64 COO min  2.000 max  2.040 mean  2.023
                                 CSR min  0.770 max  4.780 mean  4.594
                          H          min 11.835 max 11.840 mean 11.838
gen4.mtx
 Regular
                          GPU 64 COO min  4.880 max  4.980 mean  4.900
                                 CSR min 10.020 max 11.300 mean 10.716
                          H          min  9.234 max  9.234 mean  9.234
 Row-Premute
                          GPU 64 COO min  4.860 max  4.930 mean  4.890
                                 CSR min  0.330 max 11.200 mean 10.038
                          H          min 10.249 max 10.254 mean 10.252
 Row-Gradient
                          GPU 64 COO min  4.860 max  4.990 mean  4.908
                                 CSR min  9.160 max 11.240 mean 10.435
                          H          min  9.939 max  9.961 mean  9.947
 Column-Gradient
                          GPU 64 COO min  4.780 max  4.880 mean  4.816
                                 CSR min  7.770 max 10.570 mean  9.407
                          H          min 10.851 max 10.876 mean 10.864
 Row-Column-Permute
                          GPU 64 COO min  4.850 max  4.950 mean  4.886
                                 CSR min 10.220 max 11.280 mean 10.748
                          H          min 10.250 max 10.255 mean 10.252
lp_fit2d.mtx
 Regular
                          GPU 64 COO min  4.360 max  4.640 mean  4.515
                                 CSR min 10.080 max 10.900 mean 10.491
                          H          min 11.109 max 11.109 mean 11.109
 Row-Premute
                          GPU 64 COO min  4.170 max  4.630 mean  4.476
                                 CSR min  0.910 max 10.910 mean 10.257
                          H          min 11.098 max 11.104 mean 11.101
 Row-Gradient
                          GPU 64 COO min  4.370 max  4.630 mean  4.529
                                 CSR min 10.030 max 10.970 mean 10.624
                          H          min 11.109 max 11.109 mean 11.109
 Column-Gradient
                          GPU 64 COO min  4.250 max  4.640 mean  4.499
                                 CSR min  8.510 max 11.010 mean 10.505
                          H          min 11.328 max 11.333 mean 11.331
 Row-Column-Permute
                          GPU 64 COO min  4.350 max  4.640 mean  4.511
                                 CSR min 10.040 max 10.790 mean 10.468
                          H          min 11.097 max 11.106 mean 11.101
lp_osa_07.mtx
 Regular
                          GPU 64 COO min  0.460 max  3.640 mean  3.456
                                 CSR min  5.570 max  8.530 mean  8.106
                          H          min  8.412 max  8.412 mean  8.412
 Row-Premute
                          GPU 64 COO min  3.140 max  3.450 mean  3.367
                                 CSR min  7.600 max  8.070 mean  7.853
                          H          min  9.255 max  9.258 mean  9.256
 Row-Gradient
                          GPU 64 COO min  3.190 max  3.610 mean  3.509
                                 CSR min  0.000 max  8.260 mean  7.597
                          H          min  8.583 max  8.678 mean  8.670
 Column-Gradient
                          GPU 64 COO min  3.330 max  3.500 mean  3.416
                                 CSR min  6.730 max  7.540 mean  7.199
                          H          min  9.542 max  9.604 mean  9.581
 Row-Column-Permute
                          GPU 64 COO min  3.290 max  3.430 mean  3.365
                                 CSR min  7.390 max  8.060 mean  7.832
                          H          min  9.255 max  9.258 mean  9.256
Maragal_6.mtx
 Regular
                          GPU 64 COO min  4.160 max  4.310 mean  4.217
                                 CSR min  4.940 max  4.960 mean  4.956
                          H          min  9.930 max  9.930 mean  9.930
 Row-Premute
                          GPU 64 COO min  4.220 max  4.240 mean  4.225
                                 CSR min  4.750 max 13.040 mean  5.133
                          H          min 10.776 max 10.778 mean 10.777
 Row-Gradient
                          GPU 64 COO min  4.180 max  4.450 mean  4.245
                                 CSR min  4.880 max  4.940 mean  4.915
                          H          min 11.259 max 11.302 mean 11.281
 Column-Gradient
                          GPU 64 COO min  4.200 max  4.250 mean  4.236
                                 CSR min  4.800 max  4.890 mean  4.859
                          H          min 12.022 max 12.073 mean 12.051
 Row-Column-Permute
                          GPU 64 COO min  4.210 max  4.230 mean  4.222
                                 CSR min  4.860 max  4.890 mean  4.887
                          H          min 10.776 max 10.778 mean 10.778
mhd4800a.mtx
 Regular
                          GPU 64 COO min  4.570 max  4.710 mean  4.608
                                 CSR min 12.690 max 13.940 mean 13.369
                          H          min  7.132 max  7.132 mean  7.132
 Row-Premute
                          GPU 64 COO min  4.420 max  4.520 mean  4.445
                                 CSR min 10.520 max 10.880 mean 10.696
                          H          min 10.960 max 10.968 mean 10.963
 Row-Gradient
                          GPU 64 COO min  4.570 max  4.690 mean  4.605
                                 CSR min  4.550 max 13.350 mean 12.479
                          H          min  9.508 max  9.527 mean  9.520
 Column-Gradient
                          GPU 64 COO min  4.430 max  4.530 mean  4.461
                                 CSR min 10.250 max 10.940 mean 10.603
                          H          min 10.934 max 10.945 mean 10.939
 Row-Column-Permute
                          GPU 64 COO min  4.420 max  4.520 mean  4.450
                                 CSR min  7.380 max 10.900 mean 10.598
                          H          min 10.959 max 10.967 mean 10.963
mult_dcop_01.mtx
 Regular
                          GPU 64 COO min  3.420 max  3.630 mean  3.555
                                 CSR min  3.650 max  4.090 mean  3.814
                          H          min  9.689 max  9.689 mean  9.689
 Row-Premute
                          GPU 64 COO min  3.450 max  3.580 mean  3.521
                                 CSR min  3.610 max  4.150 mean  3.785
                          H          min 10.738 max 10.742 mean 10.740
 Row-Gradient
                          GPU 64 COO min  3.510 max  3.660 mean  3.579
                                 CSR min  3.650 max  4.160 mean  3.806
                          H          min 10.576 max 10.585 mean 10.580
 Column-Gradient
                          GPU 64 COO min  3.460 max  3.650 mean  3.584
                                 CSR min  3.660 max  4.240 mean  3.799
                          H          min 10.826 max 10.842 mean 10.836
 Row-Column-Permute
                          GPU 64 COO min  3.470 max  3.580 mean  3.532
                                 CSR min  3.600 max  3.980 mean  3.743
                          H          min 10.738 max 10.742 mean 10.740
mult_dcop_02.mtx
 Regular
                          GPU 64 COO min  3.390 max  3.660 mean  3.585
                                 CSR min  0.960 max  4.330 mean  4.162
                          H          min  9.689 max  9.689 mean  9.689
 Row-Premute
                          GPU 64 COO min  3.310 max  3.600 mean  3.488
                                 CSR min  0.620 max  4.290 mean  4.132
                          H          min 10.738 max 10.743 mean 10.740
 Row-Gradient
                          GPU 64 COO min  3.310 max  3.670 mean  3.593
                                 CSR min  4.130 max  4.430 mean  4.331
                          H          min 10.576 max 10.584 mean 10.580
 Column-Gradient
                          GPU 64 COO min  0.550 max  3.660 mean  3.486
                                 CSR min  3.890 max  4.410 mean  4.275
                          H          min 10.831 max 10.843 mean 10.836
 Row-Column-Permute
                          GPU 64 COO min  3.470 max  3.590 mean  3.542
                                 CSR min  4.190 max  4.290 mean  4.242
                          H          min 10.738 max 10.742 mean 10.740
mult_dcop_03.mtx
 Regular
                          GPU 64 COO min  3.360 max  3.660 mean  3.550
                                 CSR min  3.650 max  4.090 mean  3.813
                          H          min  9.689 max  9.689 mean  9.689
 Row-Premute
                          GPU 64 COO min  3.450 max  3.580 mean  3.521
                                 CSR min  3.610 max  4.160 mean  3.784
                          H          min 10.738 max 10.743 mean 10.740
 Row-Gradient
                          GPU 64 COO min  3.470 max  3.660 mean  3.572
                                 CSR min  3.640 max  4.190 mean  3.809
                          H          min 10.572 max 10.584 mean 10.580
 Column-Gradient
                          GPU 64 COO min  3.430 max  3.650 mean  3.562
                                 CSR min  3.670 max  4.290 mean  3.793
                          H          min 10.828 max 10.840 mean 10.834
 Row-Column-Permute
                          GPU 64 COO min  3.370 max  3.610 mean  3.502
                                 CSR min  3.610 max  3.970 mean  3.744
                          H          min 10.738 max 10.741 mean 10.740
OPF_3754.mtx
 Regular
                          GPU 64 COO min  4.700 max  4.930 mean  4.842
                                 CSR min  6.230 max  6.600 mean  6.411
                          H          min  8.393 max  8.393 mean  8.393
 Row-Premute
                          GPU 64 COO min  4.620 max  4.890 mean  4.787
                                 CSR min  5.780 max  6.310 mean  6.192
                          H          min 11.265 max 11.272 mean 11.269
 Row-Gradient
                          GPU 64 COO min  4.570 max  4.870 mean  4.776
                                 CSR min  5.770 max  6.510 mean  6.302
                          H          min 10.464 max 10.473 mean 10.468
 Column-Gradient
                          GPU 64 COO min  4.580 max  4.870 mean  4.756
                                 CSR min  5.630 max  6.180 mean  6.055
                          H          min 11.394 max 11.401 mean 11.397
 Row-Column-Permute
                          GPU 64 COO min  4.610 max  4.900 mean  4.780
                                 CSR min  5.010 max  6.300 mean  6.113
                          H          min 11.268 max 11.272 mean 11.270
OPF_6000.mtx
 Regular
                          GPU 64 COO min  3.780 max  3.920 mean  3.864
                                 CSR min  4.270 max  4.360 mean  4.332
                          H          min  8.799 max  8.799 mean  8.799
 Row-Premute
                          GPU 64 COO min  3.770 max  3.870 mean  3.821
                                 CSR min  3.970 max 11.050 mean  4.439
                          H          min 11.872 max 11.877 mean 11.875
 Row-Gradient
                          GPU 64 COO min  3.700 max  3.870 mean  3.795
                                 CSR min  4.330 max  4.440 mean  4.403
                          H          min 11.109 max 11.116 mean 11.113
 Column-Gradient
                          GPU 64 COO min  3.690 max  3.870 mean  3.804
                                 CSR min  4.260 max  4.340 mean  4.308
                          H          min 12.041 max 12.045 mean 12.043
 Row-Column-Permute
                          GPU 64 COO min  3.780 max  3.860 mean  3.819
                                 CSR min  4.090 max  4.290 mean  4.259
                          H          min 11.873 max 11.877 mean 11.876
shermanACb.mtx
 Regular
                          GPU 64 COO min  2.920 max  3.140 mean  3.048
                                 CSR min  5.550 max  5.980 mean  5.803
                          H          min  8.600 max  8.600 mean  8.600
 Row-Premute
                          GPU 64 COO min  2.760 max  3.020 mean  2.898
                                 CSR min  2.660 max  5.830 mean  5.632
                          H          min 10.377 max 10.381 mean 10.379
 Row-Gradient
                          GPU 64 COO min  2.800 max  3.040 mean  2.944
                                 CSR min  5.330 max  6.020 mean  5.742
                          H          min  9.919 max  9.925 mean  9.922
 Column-Gradient
                          GPU 64 COO min  2.720 max  3.010 mean  2.926
                                 CSR min  0.000 max  5.840 mean  5.513
                          H          min 10.587 max 10.596 mean 10.591
 Row-Column-Permute
                          GPU 64 COO min  2.780 max  3.030 mean  2.939
                                 CSR min  4.860 max  5.810 mean  5.667
                          H          min 10.376 max 10.382 mean 10.379
TSOPF_FS_b9_c6.mtx
 Regular
                          GPU 64 COO min  0.000 max  0.000 mean  0.000
                                 CSR min  0.000 max  0.000 mean  0.000
                          H          min  7.380 max  7.380 mean  7.380
 Row-Premute
                          GPU 64 COO min  4.540 max  4.940 mean  4.874
                                 CSR min  6.280 max  6.520 mean  6.403
                          H          min 10.042 max 10.047 mean 10.044
 Row-Gradient
                          GPU 64 COO min  4.830 max  4.930 mean  4.875
                                 CSR min  5.790 max  6.560 mean  6.289
                          H          min  9.675 max  9.706 mean  9.692
 Column-Gradient
                          GPU 64 COO min  4.790 max  4.960 mean  4.880
                                 CSR min  5.760 max  6.450 mean  6.204
                          H          min 10.601 max 10.661 mean 10.626
 Row-Column-Permute
                          GPU 64 COO min  4.330 max  4.950 mean  4.845
                                 CSR min  5.740 max  6.500 mean  6.375
                          H          min 10.041 max 10.046 mean 10.044
TSOPF_RS_b39_c7.mtx
 Regular
                          GPU 64 COO min  4.300 max  4.430 mean  4.364
                                 CSR min  4.480 max  4.750 mean  4.716
                          H          min  7.304 max  7.304 mean  7.304
 Row-Premute
                          GPU 64 COO min  4.260 max  4.400 mean  4.353
                                 CSR min  4.490 max  4.770 mean  4.734
                          H          min 10.536 max 10.541 mean 10.539
 Row-Gradient
                          GPU 64 COO min  3.970 max  4.420 mean  4.338
                                 CSR min  4.620 max  4.820 mean  4.789
                          H          min  9.638 max  9.644 mean  9.641
 Column-Gradient
                          GPU 64 COO min  4.240 max  4.430 mean  4.368
                                 CSR min  4.710 max  4.770 mean  4.736
                          H          min 11.129 max 11.222 mean 11.205
 Row-Column-Permute
                          GPU 64 COO min  4.260 max  4.410 mean  4.359
                                 CSR min  4.660 max  4.760 mean  4.738
                          H          min 10.537 max 10.541 mean 10.539
\end{verbatim}
}

\section{Fiji}
\label{sec:fiji}
{\tiny

\begin{verbatim}
mult_dcop_03.mtx
 Regular
                          GPU 64 COO min  5.140 max  5.140 mean  5.140
                                 CSR min 10.340 max 10.390 mean 10.365
                          H          min  9.689 max  9.689 mean  9.689
 Row-Premute
                          GPU 64 COO min  4.970 max  4.990 mean  4.980
                                 CSR min  9.420 max  9.430 mean  9.425
                          H          min 10.739 max 10.739 mean 10.739
 Row-Gradient
                          GPU 64 COO min  5.080 max  5.090 mean  5.085
                                 CSR min  9.720 max 10.300 mean 10.010
                          H          min 10.579 max 10.582 mean 10.580
 Column-Gradient
                          GPU 64 COO min  5.030 max  5.120 mean  5.075
                                 CSR min  9.330 max  9.770 mean  9.550
                          H          min 10.835 max 10.838 mean 10.836
 Row-Column-Permute
                          GPU 64 COO min  5.000 max  5.010 mean  5.005
                                 CSR min  7.580 max  9.460 mean  8.520
                          H          min 10.739 max 10.741 mean 10.740
mult_dcop_03.mtx
 Regular
                          GPU 64 COO min  5.140 max  5.140 mean  5.140
                                 CSR min 10.340 max 10.390 mean 10.365
                          H          min  9.689 max  9.689 mean  9.689
 Row-Premute
                          GPU 64 COO min  4.970 max  4.990 mean  4.980
                                 CSR min  9.420 max  9.430 mean  9.425
                          H          min 10.739 max 10.739 mean 10.739
 Row-Gradient
                          GPU 64 COO min  5.080 max  5.090 mean  5.085
                                 CSR min  9.720 max 10.300 mean 10.010
                          H          min 10.579 max 10.582 mean 10.580
 Column-Gradient
                          GPU 64 COO min  5.030 max  5.120 mean  5.075
                                 CSR min  9.330 max  9.770 mean  9.550
                          H          min 10.835 max 10.838 mean 10.836
 Row-Column-Permute
                          GPU 64 COO min  5.000 max  5.010 mean  5.005
                                 CSR min  7.580 max  9.460 mean  8.520
                          H          min 10.739 max 10.741 mean 10.740
mult_dcop_03.mtx
 Regular
                          GPU 64 COO min  5.130 max  5.220 mean  5.142
                                 CSR min  7.250 max  9.320 mean  7.722
                          H          min  9.689 max  9.689 mean  9.689
 Row-Premute
                          GPU 64 COO min  4.980 max  5.030 mean  4.999
                                 CSR min  6.460 max  8.470 mean  6.950
                          H          min 10.738 max 10.742 mean 10.740
 Row-Gradient
                          GPU 64 COO min  5.070 max  5.140 mean  5.088
                                 CSR min  6.780 max  8.700 mean  7.268
                          H          min 10.572 max 10.584 mean 10.580
 Column-Gradient
                          GPU 64 COO min  4.980 max  5.030 mean  5.010
                                 CSR min  6.390 max  7.640 mean  6.982
                          H          min 10.825 max 10.845 mean 10.836
 Row-Column-Permute
                          GPU 64 COO min  4.990 max  5.010 mean  4.997
                                 CSR min  6.300 max  7.160 mean  6.636
                          H          min 10.738 max 10.743 mean 10.740
mult_dcop_01.mtx
 Regular
                          GPU 64 COO min  5.120 max  5.140 mean  5.134
                                 CSR min  6.990 max  9.230 mean  7.546
                          H          min  9.689 max  9.689 mean  9.689
 Row-Premute
                          GPU 64 COO min  4.990 max  5.020 mean  5.004
                                 CSR min  6.370 max  7.220 mean  6.771
                          H          min 10.738 max 10.743 mean 10.740
 Row-Gradient
                          GPU 64 COO min  5.060 max  5.100 mean  5.082
                                 CSR min  6.730 max  7.720 mean  7.317
                          H          min 10.574 max 10.585 mean 10.580
 Column-Gradient
                          GPU 64 COO min  4.980 max  5.100 mean  5.012
                                 CSR min  6.580 max  7.510 mean  7.054
                          H          min 10.828 max 10.842 mean 10.835
 Row-Column-Permute
                          GPU 64 COO min  4.970 max  5.000 mean  4.986
                                 CSR min  6.390 max  7.050 mean  6.677
                          H          min 10.738 max 10.742 mean 10.740
mult_dcop_02.mtx
 Regular
                          GPU 64 COO min  5.120 max  5.140 mean  5.133
                                 CSR min  6.950 max  7.590 mean  7.336
                          H          min  9.689 max  9.689 mean  9.689
 Row-Premute
                          GPU 64 COO min  4.970 max  4.990 mean  4.984
                                 CSR min  6.440 max  7.110 mean  6.719
                          H          min 10.738 max 10.742 mean 10.740
 Row-Gradient
                          GPU 64 COO min  5.070 max  5.150 mean  5.086
                                 CSR min  6.650 max  7.930 mean  7.304
                          H          min 10.574 max 10.587 mean 10.580
 Column-Gradient
                          GPU 64 COO min  4.980 max  5.040 mean  5.012
                                 CSR min  6.520 max  7.650 mean  7.139
                          H          min 10.829 max 10.846 mean 10.836
 Row-Column-Permute
                          GPU 64 COO min  4.970 max  5.050 mean  4.983
                                 CSR min  6.440 max  7.380 mean  6.779
                          H          min 10.738 max 10.743 mean 10.740
lp_fit2d.mtx
 Regular
                          GPU 64 COO min  3.960 max  3.960 mean  3.960
                                 CSR min  6.360 max  7.450 mean  6.711
                          H          min 11.109 max 11.109 mean 11.109
 Row-Premute
                          GPU 64 COO min  3.950 max  3.980 mean  3.953
                                 CSR min  6.330 max  7.400 mean  6.661
                          H          min 11.098 max 11.104 mean 11.101
 Row-Gradient
                          GPU 64 COO min  3.960 max  3.980 mean  3.961
                                 CSR min  6.270 max 10.770 mean  7.017
                          H          min 11.109 max 11.109 mean 11.109
 Column-Gradient
                          GPU 64 COO min  3.940 max  3.960 mean  3.950
                                 CSR min  6.270 max  7.370 mean  6.696
                          H          min 11.329 max 11.334 mean 11.331
 Row-Column-Permute
                          GPU 64 COO min  3.950 max  3.960 mean  3.952
                                 CSR min  6.180 max  7.420 mean  6.641
                          H          min 11.098 max 11.105 mean 11.101
bloweya.mtx
 Regular
                          GPU 64 COO min  0.000 max  0.000 mean  0.000
                                 CSR min  0.000 max  0.000 mean  0.000
                          H          min  7.205 max  7.205 mean  7.205
 Row-Premute
                          GPU 64 COO min  4.020 max  4.030 mean  4.023
                                 CSR min  6.070 max  6.750 mean  6.340
                          H          min 11.025 max 11.031 mean 11.028
 Row-Gradient
                          GPU 64 COO min  4.090 max  4.160 mean  4.111
                                 CSR min  5.980 max  7.370 mean  6.678
                          H          min 10.295 max 10.304 mean 10.300
 Column-Gradient
                          GPU 64 COO min  3.980 max  4.010 mean  3.995
                                 CSR min  5.880 max  6.780 mean  6.295
                          H          min 10.881 max 10.887 mean 10.883
 Row-Column-Permute
                          GPU 64 COO min  4.020 max  4.030 mean  4.023
                                 CSR min  5.970 max  6.420 mean  6.183
                          H          min 11.025 max 11.033 mean 11.028
lp_osa_07.mtx
 Regular
                          GPU 64 COO min  4.260 max  4.270 mean  4.261
                                 CSR min  6.440 max  7.640 mean  6.863
                          H          min  8.412 max  8.412 mean  8.412
 Row-Premute
                          GPU 64 COO min  4.200 max  4.200 mean  4.200
                                 CSR min  6.020 max  7.030 mean  6.418
                          H          min  9.255 max  9.257 mean  9.256
 Row-Gradient
                          GPU 64 COO min  4.210 max  4.240 mean  4.226
                                 CSR min  6.070 max 10.050 mean  6.498
                          H          min  8.607 max  8.678 mean  8.671
 Column-Gradient
                          GPU 64 COO min  4.170 max  4.190 mean  4.180
                                 CSR min  5.610 max  7.300 mean  5.988
                          H          min  9.534 max  9.601 mean  9.585
 Row-Column-Permute
                          GPU 64 COO min  4.190 max  4.190 mean  4.190
                                 CSR min  6.070 max  7.000 mean  6.386
                          H          min  9.255 max  9.257 mean  9.256
ex19.mtx
 Regular
                          GPU 64 COO min  6.140 max  6.180 mean  6.159
                                 CSR min 12.780 max 14.400 mean 13.328
                          H          min  8.228 max  8.228 mean  8.228
 Row-Premute
                          GPU 64 COO min  5.820 max  5.850 mean  5.833
                                 CSR min  9.870 max 11.070 mean 10.372
                          H          min 11.836 max 11.840 mean 11.838
 Row-Gradient
                          GPU 64 COO min  6.070 max  6.120 mean  6.104
                                 CSR min 11.290 max 12.760 mean 12.088
                          H          min 10.743 max 10.752 mean 10.748
 Column-Gradient
                          GPU 64 COO min  5.760 max  5.840 mean  5.813
                                 CSR min  9.710 max 14.220 mean 10.376
                          H          min 11.873 max 11.882 mean 11.878
 Row-Column-Permute
                          GPU 64 COO min  5.810 max  5.860 mean  5.838
                                 CSR min  9.920 max 10.820 mean 10.240
                          H          min 11.836 max 11.841 mean 11.838
brainpc2.mtx
 Regular
                          GPU 64 COO min  0.000 max  0.000 mean  0.000
                                 CSR min  0.000 max  0.000 mean  0.000
                          H          min  7.478 max  7.478 mean  7.478
 Row-Premute
                          GPU 64 COO min  4.760 max  4.790 mean  4.773
                                 CSR min  6.930 max  7.780 mean  7.310
                          H          min  9.810 max  9.813 mean  9.811
 Row-Gradient
                          GPU 64 COO min  4.820 max  4.840 mean  4.831
                                 CSR min  7.220 max  8.290 mean  7.583
                          H          min  9.721 max  9.725 mean  9.723
 Column-Gradient
                          GPU 64 COO min  4.760 max  4.820 mean  4.779
                                 CSR min  6.870 max  8.300 mean  7.393
                          H          min 10.368 max 10.373 mean 10.370
 Row-Column-Permute
                          GPU 64 COO min  4.750 max  4.780 mean  4.765
                                 CSR min  6.940 max  7.580 mean  7.298
                          H          min  9.809 max  9.814 mean  9.811
shermanACb.mtx
 Regular
                          GPU 64 COO min  4.090 max  4.130 mean  4.112
                                 CSR min  6.320 max  7.200 mean  6.779
                          H          min  8.600 max  8.600 mean  8.600
 Row-Premute
                          GPU 64 COO min  4.020 max  4.050 mean  4.036
                                 CSR min  5.670 max  6.460 mean  6.014
                          H          min 10.376 max 10.382 mean 10.379
 Row-Gradient
                          GPU 64 COO min  4.050 max  4.100 mean  4.074
                                 CSR min  5.580 max  6.420 mean  5.996
                          H          min  9.918 max  9.924 mean  9.921
 Column-Gradient
                          GPU 64 COO min  4.010 max  4.080 mean  4.033
                                 CSR min  0.000 max  6.320 mean  5.527
                          H          min 10.543 max 10.595 mean 10.589
 Row-Column-Permute
                          GPU 64 COO min  4.020 max  4.050 mean  4.036
                                 CSR min  5.670 max  6.510 mean  6.092
                          H          min 10.377 max 10.381 mean 10.379
cvxqp3.mtx
 Regular
                          GPU 64 COO min  3.500 max  3.540 mean  3.501
                                 CSR min 11.860 max 13.100 mean 12.694
                          H          min  8.646 max  8.646 mean  8.646
 Row-Premute
                          GPU 64 COO min  3.360 max  3.370 mean  3.365
                                 CSR min  6.210 max  7.610 mean  6.631
                          H          min 11.027 max 11.032 mean 11.030
 Row-Gradient
                          GPU 64 COO min  3.370 max  3.380 mean  3.376
                                 CSR min  6.170 max  7.070 mean  6.499
                          H          min 11.059 max 11.068 mean 11.064
 Column-Gradient
                          GPU 64 COO min  3.350 max  3.390 mean  3.371
                                 CSR min  6.150 max  7.180 mean  6.531
                          H          min 11.125 max 11.133 mean 11.130
 Row-Column-Permute
                          GPU 64 COO min  3.350 max  3.380 mean  3.364
                                 CSR min  6.040 max  7.440 mean  6.603
                          H          min 11.028 max 11.033 mean 11.030
case9.mtx
 Regular
                          GPU 64 COO min  0.000 max  0.000 mean  0.000
                                 CSR min  0.000 max  0.000 mean  0.000
                          H          min  7.380 max  7.380 mean  7.380
 Row-Premute
                          GPU 64 COO min  4.130 max  4.170 mean  4.134
                                 CSR min  6.180 max  9.200 mean  6.796
                          H          min 10.041 max 10.046 mean 10.044
 Row-Gradient
                          GPU 64 COO min  4.150 max  4.220 mean  4.163
                                 CSR min  6.410 max  7.500 mean  6.816
                          H          min  9.682 max  9.706 mean  9.693
 Column-Gradient
                          GPU 64 COO min  4.080 max  4.110 mean  4.096
                                 CSR min  6.020 max  7.220 mean  6.309
                          H          min 10.597 max 10.658 mean 10.631
 Row-Column-Permute
                          GPU 64 COO min  4.120 max  4.140 mean  4.130
                                 CSR min  6.210 max  7.200 mean  6.609
                          H          min 10.041 max 10.046 mean 10.044
TSOPF_FS_b9_c6.mtx
 Regular
                          GPU 64 COO min  0.000 max  0.000 mean  0.000
                                 CSR min  0.000 max  0.000 mean  0.000
                          H          min  7.380 max  7.380 mean  7.380
 Row-Premute
                          GPU 64 COO min  4.120 max  4.140 mean  4.129
                                 CSR min  6.170 max  7.160 mean  6.664
                          H          min 10.041 max 10.045 mean 10.043
 Row-Gradient
                          GPU 64 COO min  4.150 max  4.180 mean  4.162
                                 CSR min  6.420 max  7.360 mean  6.723
                          H          min  9.682 max  9.706 mean  9.693
 Column-Gradient
                          GPU 64 COO min  4.080 max  4.120 mean  4.096
                                 CSR min  5.880 max  7.090 mean  6.403
                          H          min 10.611 max 10.660 mean 10.637
 Row-Column-Permute
                          GPU 64 COO min  4.130 max  4.140 mean  4.130
                                 CSR min  6.330 max  7.390 mean  6.695
                          H          min 10.042 max 10.047 mean 10.044
OPF_6000.mtx
 Regular
                          GPU 64 COO min  7.270 max  7.370 mean  7.293
                                 CSR min 12.890 max 14.500 mean 13.566
                          H          min  8.799 max  8.799 mean  8.799
 Row-Premute
                          GPU 64 COO min  6.640 max  6.720 mean  6.678
                                 CSR min  9.680 max 11.600 mean 10.040
                          H          min 11.873 max 11.877 mean 11.875
 Row-Gradient
                          GPU 64 COO min  7.090 max  7.140 mean  7.122
                                 CSR min 11.250 max 13.030 mean 12.142
                          H          min 11.110 max 11.117 mean 11.114
 Column-Gradient
                          GPU 64 COO min  6.590 max  6.710 mean  6.644
                                 CSR min  9.400 max 13.140 mean  9.991
                          H          min 12.040 max 12.046 mean 12.043
 Row-Column-Permute
                          GPU 64 COO min  6.640 max  6.710 mean  6.679
                                 CSR min  9.690 max 10.740 mean 10.050
                          H          min 11.874 max 11.877 mean 11.875
OPF_3754.mtx
 Regular
                          GPU 64 COO min  4.430 max  4.450 mean  4.443
                                 CSR min  9.710 max 13.000 mean 11.377
                          H          min  8.393 max  8.393 mean  8.393
 Row-Premute
                          GPU 64 COO min  4.230 max  4.250 mean  4.240
                                 CSR min  7.430 max  8.750 mean  7.986
                          H          min 11.266 max 11.272 mean 11.269
 Row-Gradient
                          GPU 64 COO min  4.370 max  4.420 mean  4.382
                                 CSR min  8.160 max  9.470 mean  8.682
                          H          min 10.462 max 10.473 mean 10.468
 Column-Gradient
                          GPU 64 COO min  4.210 max  4.240 mean  4.227
                                 CSR min  7.160 max  8.080 mean  7.595
                          H          min 11.394 max 11.401 mean 11.398
 Row-Column-Permute
                          GPU 64 COO min  4.230 max  4.250 mean  4.243
                                 CSR min  7.230 max  8.940 mean  8.056
                          H          min 11.264 max 11.271 mean 11.269
c-47.mtx
 Regular
                          GPU 64 COO min  5.320 max  5.340 mean  5.329
                                 CSR min  8.890 max  9.590 mean  9.249
                          H          min  8.364 max  8.364 mean  8.364
 Row-Premute
                          GPU 64 COO min  5.240 max  5.250 mean  5.241
                                 CSR min  7.790 max  8.890 mean  8.214
                          H          min 10.059 max 10.063 mean 10.061
 Row-Gradient
                          GPU 64 COO min  5.230 max  5.260 mean  5.242
                                 CSR min  7.080 max  8.050 mean  7.673
                          H          min 10.206 max 10.226 mean 10.218
 Column-Gradient
                          GPU 64 COO min  5.080 max  5.120 mean  5.105
                                 CSR min  5.780 max  6.970 mean  6.359
                          H          min 11.205 max 11.233 mean 11.222
 Row-Column-Permute
                          GPU 64 COO min  5.220 max  5.250 mean  5.227
                                 CSR min  7.860 max  8.710 mean  8.247
                          H          min 10.059 max 10.064 mean 10.061
mhd4800a.mtx
 Regular
                          GPU 64 COO min  3.090 max  3.100 mean  3.098
                                 CSR min 11.570 max 12.290 mean 12.092
                          H          min  7.132 max  7.132 mean  7.132
 Row-Premute
                          GPU 64 COO min  3.020 max  3.020 mean  3.020
                                 CSR min  5.560 max  7.270 mean  6.007
                          H          min 10.959 max 10.968 mean 10.963
 Row-Gradient
                          GPU 64 COO min  3.080 max  3.100 mean  3.088
                                 CSR min 10.250 max 12.150 mean 11.340
                          H          min  9.509 max  9.528 mean  9.520
 Column-Gradient
                          GPU 64 COO min  3.020 max  3.050 mean  3.026
                                 CSR min  5.530 max 10.580 mean  6.432
                          H          min 10.933 max 10.946 mean 10.939
 Row-Column-Permute
                          GPU 64 COO min  3.020 max  3.020 mean  3.020
                                 CSR min  5.510 max  6.830 mean  6.136
                          H          min 10.959 max 10.967 mean 10.963
gen4.mtx
 Regular
                          GPU 64 COO min  3.300 max  3.320 mean  3.308
                                 CSR min  5.250 max  6.340 mean  5.705
                          H          min  9.234 max  9.234 mean  9.234
 Row-Premute
                          GPU 64 COO min  3.290 max  3.310 mean  3.299
                                 CSR min  5.190 max  7.420 mean  5.683
                          H          min 10.249 max 10.254 mean 10.252
 Row-Gradient
                          GPU 64 COO min  3.300 max  3.310 mean  3.301
                                 CSR min  5.370 max  6.310 mean  5.659
                          H          min  9.934 max  9.958 mean  9.948
 Column-Gradient
                          GPU 64 COO min  3.240 max  3.260 mean  3.249
                                 CSR min  5.090 max  8.660 mean  5.546
                          H          min 10.853 max 10.873 mean 10.864
 Row-Column-Permute
                          GPU 64 COO min  3.290 max  3.320 mean  3.296
                                 CSR min  5.190 max  7.550 mean  5.659
                          H          min 10.249 max 10.255 mean 10.252
Maragal_6.mtx
 Regular
                          GPU 64 COO min 10.580 max 10.620 mean 10.599
                                 CSR min 15.620 max 16.470 mean 15.832
                          H          min  9.930 max  9.930 mean  9.930
 Row-Premute
                          GPU 64 COO min 10.340 max 10.430 mean 10.362
                                 CSR min 12.880 max 13.340 mean 13.057
                          H          min 10.777 max 10.778 mean 10.777
 Row-Gradient
                          GPU 64 COO min 10.650 max 10.740 mean 10.688
                                 CSR min 12.310 max 13.670 mean 12.562
                          H          min 11.247 max 11.300 mean 11.281
 Column-Gradient
                          GPU 64 COO min 10.340 max 10.440 mean 10.398
                                 CSR min  9.480 max 10.110 mean  9.782
                          H          min 12.023 max 12.069 mean 12.047
 Row-Column-Permute
                          GPU 64 COO min 10.330 max 10.380 mean 10.356
                                 CSR min 12.840 max 13.530 mean 13.119
                          H          min 10.776 max 10.778 mean 10.777
aft01.mtx
 Regular
                          GPU 64 COO min  3.680 max  3.690 mean  3.688
                                 CSR min 13.860 max 14.830 mean 14.560
                          H          min  7.811 max  7.811 mean  7.811
 Row-Premute
                          GPU 64 COO min  3.510 max  3.530 mean  3.513
                                 CSR min  6.420 max 10.520 mean  7.265
                          H          min 11.161 max 11.170 mean 11.165
 Row-Gradient
                          GPU 64 COO min  3.630 max  3.670 mean  3.643
                                 CSR min 10.760 max 13.510 mean 12.199
                          H          min 10.248 max 10.265 mean 10.258
 Column-Gradient
                          GPU 64 COO min  3.510 max  3.520 mean  3.519
                                 CSR min  6.490 max 11.230 mean  7.645
                          H          min 11.112 max 11.121 mean 11.117
 Row-Column-Permute
                          GPU 64 COO min  3.510 max  3.540 mean  3.515
                                 CSR min  6.510 max 11.650 mean  7.311
                          H          min 11.161 max 11.168 mean 11.165
TSOPF_RS_b39_c7.mtx
 Regular
                          GPU 64 COO min  5.970 max  6.010 mean  5.988
                                 CSR min 12.470 max 21.120 mean 13.816
                          H          min  7.304 max  7.304 mean  7.304
 Row-Premute
                          GPU 64 COO min  5.840 max  5.870 mean  5.856
                                 CSR min 10.780 max 15.810 mean 11.425
                          H          min 10.537 max 10.540 mean 10.539
 Row-Gradient
                          GPU 64 COO min  5.950 max  6.000 mean  5.975
                                 CSR min 11.520 max 17.250 mean 12.799
                          H          min  9.638 max  9.646 mean  9.641
 Column-Gradient
                          GPU 64 COO min  5.790 max  5.860 mean  5.827
                                 CSR min 10.500 max 14.080 mean 11.237
                          H          min 11.128 max 11.223 mean 11.209
 Row-Column-Permute
                          GPU 64 COO min  5.850 max  5.870 mean  5.855
                                 CSR min 10.790 max 15.250 mean 11.718
                          H          min 10.537 max 10.541 mean 10.539
mult_dcop_03.mtx
 Regular
                          GPU 64 COO min  5.130 max  5.220 mean  5.142
                                 CSR min  7.250 max  9.320 mean  7.722
                          H          min  9.689 max  9.689 mean  9.689
 Row-Premute
                          GPU 64 COO min  4.980 max  5.030 mean  4.999
                                 CSR min  6.460 max  8.470 mean  6.950
                          H          min 10.738 max 10.742 mean 10.740
 Row-Gradient
                          GPU 64 COO min  5.070 max  5.140 mean  5.088
                                 CSR min  6.780 max  8.700 mean  7.268
                          H          min 10.572 max 10.584 mean 10.580
 Column-Gradient
                          GPU 64 COO min  4.980 max  5.030 mean  5.010
                                 CSR min  6.390 max  7.640 mean  6.982
                          H          min 10.825 max 10.845 mean 10.836
 Row-Column-Permute
                          GPU 64 COO min  4.990 max  5.010 mean  4.997
                                 CSR min  6.300 max  7.160 mean  6.636
                          H          min 10.738 max 10.743 mean 10.740
mult_dcop_01.mtx
 Entropy
                          GPU 64 COO min  5.120 max  5.140 mean  5.134
                                 CSR min  6.990 max  9.230 mean  7.546
                          H          min  9.689 max  9.689 mean  9.689
\end{verbatim}
}




%\input{conclusion.tex}

%%%%%%%%% -- BIB STYLE AND FILE -- %%%%%%%%
\bibliographystyle{ACM-Reference-Format} \bibliography{ref}
%%%%%%%%%%%%%%%%%%%%%%%%%%%%%%%%%%%%

%\appendix{Review and Response}
%\input{review.tex}
\end{document}




  


  




  


